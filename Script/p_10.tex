\chapter{Templates}
\label{p:10}
Der englische  Begriff \emph {template} bedeutet auf deutsch \emph{Schablone}.
Eine Schablone wird bei der Produktherstellung dazu verwendet die gleiche Form immer wieder herzustellen
(oder zumindest anzureißen), gleich ob das Material nun Metall, Karton oder Holz ist.

Unser Produkt sind Funktionen und Klassen mit den Datentypen als Material.
%
\section{Template-Funktionen}
\label{p:10.1}
% V_14/v_8b

%
In irgendeinem Stadium der Implementierung stellt man pl"otzlich fest, da"s
die gleichen Funktionen f"ur verschiedene Datentypen gebraucht werden. In nachfolgendem,
einfachen Beispiel sind dies das Maximium zweier Zahlen und das Maximum eines Vektors.
%
%\exfile{A7/fkt.hpp}.
\begin{lstlisting}[caption={Header:Funktion f\"ur jeden Datentyp einzeln.},label=lst:fkt_no_template_h,
basicstyle=\scriptsize,numbers=left, numberstyle=\tiny, stepnumber=2, numbersep=5pt]
{
    int    mymax (const int    a, const int    b);
    float  mymax (const float  a, const float  b);
    double mymax (const double a, const double b);

    int    mymax_elem(const vector<int>&    x);
    float  mymax_elem(const vector<float>&  x);
    double mymax_elem(const vector<double>& x);
}
\end{lstlisting}
%

Die Implementierungen beider Funktionsgruppen unterscheiden sich
nur im jeweiligen Datentyp, die Struktur der Implementierung ist ansonsten
komplett identisch.
%\exfile{A7/fkt.cpp}.
\begin{lstlisting}[caption={Source:Funktion für jeden Datentyp einzeln.},label=lst:fkt_no_template_s,
basicstyle=\scriptsize,numbers=left, numberstyle=\tiny, stepnumber=2, numbersep=5pt]
     int    mymax (const int a, const int b)
     {
        return  a > b  ?  a  : b;
     }

     float  mymax (const float a, const float b)
     {
        return  a > b  ?  a  : b;
     }

     double mymax (const double a, const double b)
     {
        return  a > b  ?  a  : b;
     }

     float  max_elem(const vector<float>&  x)
     {
      const int=x.size();
      assert(n>0);                  // --> richtiges Exeption-handling muss rein

      float vmax = x[0];
      for (int i=1; i<n; i++)
       {
         vmax = mymax(vmax,x[i]);   // Nutze die Funktion mymax
       }

      return vmax;
     }
//       usw.
     ...
\end{lstlisting}
%%
Neben dem unguten Gef"uhl, sich ungeschickt anzustellen, m"ussen algorithmische
Verbesserungen in jeder Funktion einer Funktionsgruppe implementiert werden.
Die ist wiederum fehleranf"allig und der Gesamtcode schwerer zu warten.

\subsection{Implementierung eines Funktions-Templates}
\label{p:10.1.1}
C++ bietet die M"oglichkeit, mit Hilfe von \emph{Templates}\index{Template} (engl.: Schablonen)
eine parametrisierte Familie verwandter Funktionen zu definieren.
Ein Funktions-Template\index{Template!Funktions-} legt die Anweisungen
einer Funktion fest, wobei statt eines konkreten Typs ein Parameter verwendet wird
\cite[p.365]{KirchPrinz:2002:OOP}.
Vorteile dieser Templates sind:
\begin{itemize}
 \item Ein Funktionen-Template mu"s nur einmal kodiert werden.
 \item Einzelne Funktionen zu einem konkreten Parameter werden anhand des Templates
 	automatisch erzeugt.
 \item Typunabh"angige Bestandteile (der Algorithmus) k"onnen ausgetestet werden und
 	funktionieren dann f"ur die anderen Parametertypen.
 \item Es besteht immer noch die M"oglichkeit, f"ur bestimmte Typen spezielle L"osungen
 	anzugeben.
\end{itemize}

Einem Funktions-Template wird das Pr"afix
\\[0.5ex]
\verb|template <class T>|
\\[0.5ex]
vorangestellt. Der Parameter \verb| T | ist hierbei der Typname welcher in der nachfolgenden
Definition benutzt wird.
Dabei schlie"st das Schl"usselwort \verb|class| auch einfache Datentypen wie
\verb|int| oder \verb|double| ein.

Unsere Funktionsfamilien werden nunmehr mit Templates so definiert und in ein File \emph{tfkt.tpp}
geschrieben.
%\exfile{A7/tfkt.cpp}.\label{code:A7tfkt.cpp}
%
\begin{lstlisting}[caption={Source: Templatefunktion.},label=lst:fkt_template_s,
basicstyle=\scriptsize,numbers=left, numberstyle=\tiny, stepnumber=2, numbersep=5pt]
{
//				tfkt.tpp
#include <vector>
#include <cassert>
using namespace std;
//#include "tfkt.hpp"

template <class T>
T  mymax (const T  a,const T  b)
 {
  return  a > b  ?  a  : b;
 }

template <class T>
T  max_elem(const vector<T>&  x)
 {
  const int=x.size();
  assert(n>0);		// --> richtiges Exeption-handling muss rein
  T vmax = x[0];
  for (int i=1; i<n; i++)
   {
     vmax = mymax(vmax,x[i]);
   }

  return vmax;
 }
}
\end{lstlisting}
%
Die Deklaration ist dann im Headerfile \emph{tfkt.h}
%\exfile{A7/tfkt.hpp}.
\begin{lstlisting}[caption={Header: Templatefunktion.},label=lst:fkt_template_h,
basicstyle=\scriptsize,numbers=left, numberstyle=\tiny, stepnumber=2, numbersep=5pt]
#ifndef FILE_TFKT
#define FILE_TFKT
#include <vector>
using namespace std;

    template <class T>
    T  mymax (const T  a, const T  b);

    template <class T>
    T  max_elem(const const vector<T>&  x);

#include "tfkt.tpp"                   // Include the sources in case of templates!!
#endif
\end{lstlisting}
%
Da in der STL eine Funktionsfamilie \verb|max| bereits vorhanden ist, hei"st unsere
Funktionsfamilie \verb|mymax|. Ansonsten ergebens sich Zweideutigkeiten bei der
Auswahl der richtigen Funktion.

Die Anwendung unserer Funktionstemplates im
Hauptprogramm
%\exfile{A7/main.cpp}
\label{code:A7main.cpp}
ist relativ einfach.

\begin{lstlisting}[caption={Templatefunktion anwenden.},label=lst:fkt_template_main,
basicstyle=\scriptsize,numbers=left, numberstyle=\tiny, stepnumber=2, numbersep=5pt]
#include <iostream>
#include <vector>
#include "tfkt.hpp"                // Header mit Templates und deren Sources
using namespace std;

int main()
{
 const int N = 10;
 vector<int>    ia(N);
 vector<float>  fa(N);
 vector<double> da(N);

 for (int i=0; i<N; i++)           // Vektoren initialisieren
  { ia[i] = i;
    fa[i] = i/(i+1.0f);
    da[i] = i/(i+1.0);
  }

 int    im = max_elem(ia);         // int
 cout << "   int-Feld(max):  " << im << endl;

 float  fm = max_elem(fa);         // float
 cout << " float-Feld(max):  " << fm << endl;

 double dm = max_elem(da);         // double
 cout << "double-Feld(max):  " << dm << endl;

 return 0;
}
\end{lstlisting}

Ein Funktions-Template kann auch mit mehreren Typparametern definiert werden.
%			Energienorm fr Vektor ||v||_A
%caption={Templatefunktion anwenden.},label=lst:fkt_template_main,
\begin{lstlisting}[
basicstyle=\scriptsize,numbers=left, numberstyle=\tiny, stepnumber=2, numbersep=5pt]
    template<class A, class B>
    B& func(const int n, const A mm, vector<B>& v)
    {
    ...
    }
\end{lstlisting}
%
%\subsection{Das Schl"usselwort \texttt{export}}
%\label{sec:A7.3}
%%
%Zeile~13 im File \textit{tfkt.hpp} inkludiert ungew"ohnlicherweise die
%Definitionen unserer Funktions-Templates. Kommentiert man diese Zeilen aus,
%so kann der \verb|g++|-Compiler (und "ahnlich der \verb|icc|) zwar
%compileren, jedoch nicht linken.\index{export}
%\begin{verbatim}
%LINUX> g++ -c tfkt.cpp
%LINUX> g++ -c main.cpp
%LINUX> g++ -o main main.o tfkt.o

%main.o(.text+0x78): In function `main':
%: undefined reference to `int max_elem<int>(int, int const*)'
%main.o(.text+0xc6): In function `main':
%: undefined reference to `float max_elem<float>(int, float const*)'
%main.o(.text+0x11a): In function `main':
%: undefined reference to `double max_elem<double>(int, double const*)'
%main.o(.text+0x17a): In function `main':
%: undefined reference to `float max<float>(float&, float&)'
%\end{verbatim}
%Der Grund daf"ur ist, da"s der Compiler den Code nur dann
%compilieren kann, wenn er ihn vollst"andig sieht. Beim Compilieren von
%\textit{tfkt.cpp} konnte er noch nicht wissen, f"ur welchen Typ die Funktionen
%definiert werden sollen und beim Compilieren von \textit{main.cpp} ist die
%Implementierung der Funktions-Templates unbekannt. Es wurde also "uberhaupt keine
%Funktionen \verb|max| oder \verb|max_elem| compiliert. Aus diesem Grunde ist die
%Includedeklaration in Zeile~13 des Files \textit{tfkt.hpp} notwendig.

%W"unschenswert w"are nat"urlich auch  bei Templates eine saubere Trennung
%zwischen Header- und Quelltextfile, bei dem ein Funktions-Template trotzdem
%global verf"ugbar ist.
%Laut C++-Standard wird genau dies durch das Schl"usselwort \verb|export|
%bewirkt~\cite[p.248f]{Yang:2001:COO}.
%Soweit die Theorie - in der Praxis erh"alt man
%allerdings die Compilerwarnung\\
%\verb|tfkt.cpp:11: warning: keyword `export' not implemented, and will be ignored|
%\\
%d.h., diese w"unschenswerte Funktionalit"at von C++ ist derzeit noch nicht verf"ugbar
%(J"anner 2008, nach meinem Wissen).
%Da diese Funktionalit"at aber fr"uher oder sp"ater verf"ugbar sein wird, sollte man
%erstmal die Quelltextfiles in die Headerfiles inkludieren und
%entsprechende Funktions-Templates jetzt schon mit \verb|export| im
%Quelltextfile kennzeichnen
%(Zeilen 5 und 11 im Code auf Seite~\pageref{code:A7tfkt.cpp}).
%Falls das \verb|export|- Schl"usselwort eines Tages korrekt unterst"utzt wird, dann
%braucht man nur im Headerfile das Inkludieren des Quelltextfiles
%auszukommentieren \cite[p.447f]{Schmaranz:2002:SCP}.
%%
%
\subsection{Implizite und explizite Templateargumente}
\label{p:10.1.2}
%
Im Hauptprogramm auf Seite~\pageref{code:A7main.cpp} wurde das richtige
Templateargument \verb| T | , und damit die konkrete Funktion,
anhand der Funktionsparameter bestimmt.
Falls dies nicht eindeutig ist, bzw. das Templateargument nicht in der
Parameterliste der Funktion vorkommt, dann mu"s das Templateargument
beim Funktionsaufruf explizit angegeben werden. Das folgene Codefragment
demonstriert implizite und explizite Templateargumente in den Zeilen~12 und~16.

\begin{lstlisting}[caption={Templatefunktion anwenden.},label=lst:fkt_template_main_2,
basicstyle=\scriptsize,numbers=left, numberstyle=\tiny, stepnumber=2, numbersep=5pt]
#include <iostream>
#include <vector>
#include "tfkt.hpp"                // Header mit Templates und deren Sources
using namespace std;

int main()
{
 const int N = 10;
 vector<double> da(N);
 ....                              // Vektoren initialisieren

 double dm = max_elem(da);         // double, implicit instantiation

 float  a=5.455, fm;
 double b=-3.344;
 fm = mymax<float>(a,b);           // float,  explicit instantiation

 return 0;
}
\end{lstlisting}


Weitere Hinweise zu Typanpassungen bei Funktions-Templates sind in
\cite[p.371, p.377]{KirchPrinz:2002:OOP} zu finden.
Die implizite Typanpassung kann (auch zu Debuggingzwecken) beim \verb|g++| mittels der Option
\verb|-fno-implicit-templates| unterbunden werden.
%
%
\subsection{Spezialisierung}
\label{p:10.1.3}
%
So sch"on unsere allgemeinen Funktions-Templates sind, sie sind
nicht f"ur alle denkbaren Typen einsetzbar:
\begin{itemize}
 \item Das Funktions-Template enth"alt Anweisungen, die f"ur bestimmte Typen
 	nicht ausgef"uhrt werden k"onnen.
 \item Die allgemeine L"osung, die das Template bereitstellt, liefert kein
 	sinnvolles Ergebnis.
 \item F"ur bestimmte Typen gibt es bessere (schnellere, speicherschonendere) L"osungen.
\end{itemize}
F"ur solche F"alle l"a"st sich z.B., f"ur unsere Funktion \verb|mymax| eine
Spezialisierung\index{Spezialisierung} f"ur C++-Strings implementieren.


Folgender Code
%\index{A7/main2.cpp}
funktioniert mit unserer Templatefunktion \verb|mymax<T>| auch ohne Spezialisierung,
da \verb|operator>| f"ur die Klasse \verb|string| implementiert ist und einen lexikographischen
Vergleich durchf"uhrt.
\begin{lstlisting}[caption={Templatefunktion ohne Spezialisierung anwenden.},label=lst:fkt_template_spec_1,
basicstyle=\scriptsize,numbers=left, numberstyle=\tiny, stepnumber=2, numbersep=5pt]
#include <iostream>
#include <vector>
#include <string>
#include "tfkt.hpp"                // Header mit Templates und deren Sources
using namespace std;

int main()
{
  const int N = 10;
  vector<string> da(N);
  ....                             // Vektor of strings initialisieren

  max_str = mymax(name1,name2);	   // lexicographic comparison
  cout << max_str << endl;

  max_str = mymax(a[5],a[9]);      // lexicographic comparison
  cout << max_str << endl;

  max_str = max_elem(N, a);        // lexicographic comparison
  cout << max_str << endl;

 return 0;
}
\end{lstlisting}

Will man den resultierenden lexikographischen Vergleich \verb|operator>(string, string)|  in \verb|mymax| durch
einen Vergleich der Stringl"angen ersetzen, dann hilft folgende Spezialisierung:
\begin{lstlisting}[caption={Spezialisierung einer Templatefunktion.},label=lst:fkt_template_spec_2,
basicstyle=\scriptsize,numbers=left, numberstyle=\tiny, stepnumber=2, numbersep=5pt]
template<>                       // Einleitung der Spezialisierung
string mymax(const string &s1, const string &s2 )
{
  if ( s1.size()>s2.size() ) return s1;
  else                       return s2;
}
\end{lstlisting}
%
Diese Funktion "uberl"adt, bei (genau!) passenden Parametern das entsprechende
Funktions-Template von \verb|mymax|.
Das Funktions-Template \verb|max_elem| kann nun sogar mit unserer Spezialisierung
arbeiten.

%Das Spezialisierungsbeispiel aus \cite[p.372f]{KirchPrinz:2002:OOP} konnte
%ich nicht in der originalen Form zum korrekten Programmlauf "uberreden.
%Vielleicht interpretieren andere Compiler die impliziten Typkonvertierungen
%etwas anders als der \verb|g++|-3.2.3.

%
% Motivation \cite[p.365]{KirchPrinz:2002:OOP}
%
% Typ"ubereinstimmung \cite[p.371]{KirchPrinz:2002:OOP}
%
% Spezialisierung \cite[p.373]{KirchPrinz:2002:OOP}, Schmaranz, p.432
%
% Implizite und explizite Templateargumente \cite[p.377]{KirchPrinz:2002:OOP}
% \verb|erg = max(x,y)|  vs. \verb|erg = max<float>(x,y)| .
% \\
%
%
% Idee: CG_alg mit <class MATRIX, class VEKTOR> und spezialisiertem Matrix-Vektor-Produkt
% ==> Aufgabe: Matrixklassenhier. vs. Template (kombination von beiden?), Effizienz
%
%
%                     K L A S S E N
%
%
\section{Template-Klassen}
\label{p:10.2}
% V_14/v_8c
\subsection{Ein Klassen-Template f"ur \texttt{Komplex}}
\label{p:10.2.1}
%
Analog zu den Funktions-Templates bietet C++ die M"oglichkeit, Klassen-Templates zu
definieren. Die Klassen-Templates werden in Anh"angigkeit von einem noch
festzulegendem Typ (oder mehreren Typen) konstruiert.
Diese Klassen-Templates werden h"aufig bei der Erstellung von Klassenbibliotheken
eingesetzt, wir werden im Zusammenhang mit der STL in~\S\ref{p:11} auf solche Klassenbibliotheken zugreifen.

Einem Klassen-Template ist der Pr"afix \verb| template<class T> | vorangestellt, dem
die eigentliche Klassendefinition folgt.\index{Template!Klassen-}
%
\begin{lstlisting}[basicstyle=\scriptsize,numbers=left, numberstyle=\tiny, stepnumber=2, numbersep=5pt]
template<class T>
class X
{
...			// Definition der Klasse X<T>
};
\end{lstlisting}
%
Der Name des Template f"ur Klassen ist~\verb|X<T>|. Der Parameter~\texttt{T} steht wieder
f"ur einen beliebigen (auch einfachen) Datentyp. Sowohl~\texttt{T} als auch
\texttt{X<T>} werden in der Klassendefinition wie normale Datentypen verwendet.

Zur Demonstration leiten wir aus der  Klasse \texttt{Komplex}
von Seite~\pageref{lst:komplex_komplex.h_a} das Klassen-Template \texttt{Komplex<T>}
ab
%\exfile{A8/tmyvector.hpp}
indem wir den original verwandten Datentyp \texttt{double} f"ur die
Vektorelemente durch~\texttt{T} und \texttt{Komplex} durch \texttt{Komplex<T>} ersetzen.
%
\includecode[linerange={11-14,16-16,19-19,21-21,26-29,33-36,41-44,48-51,57-57,64-85,103-103},label=lst:komplex.h]{v_8c/komplex.h}{Header der Templateklasse}

Die Deklaration der Methoden erfolgt dann analog, hier sei exemplarisch
der Plus-Operator pr"asentiert im File.
%
\includecode[linerange={23-38},label=lst:komplex.tpp]{v_8c/komplex.tpp}{Implementierung der Methode einer Templateklasse}
Wichtig ist, da"s vor jeder Methodendefinition der Pr"afix \verb|template<class T>| steht.

\subsection{Mehrere Parameter}
\label{p:10.2.2}
Analog zu den Funktions-Templates k"onnen auch Klassen-Templates mit mehreren
Typparametern definiert werden.
%
\begin{lstlisting}[
basicstyle=\scriptsize,numbers=left, numberstyle=\tiny, stepnumber=2, numbersep=5pt]
    template<class T1, class T2>
    class X
    {
    ...			// Definition der Klasse X<T1,T2>
    };
\end{lstlisting}

Desweiteren ist ein Template mit einem festgelegtem Parameter, einem Argument,
m"oglich~\cite[p.397ff]{KirchPrinz:2002:OOP},
%
\begin{lstlisting}[
basicstyle=\scriptsize,numbers=left, numberstyle=\tiny, stepnumber=2, numbersep=5pt]
    template<class T, int n>
    class Queue {...};
\end{lstlisting}
mit welchem \texttt{n}, z.B., die Gr"o"se eines immer wieder gebrauchten,
internen tempor"aren Feldes bezeichnet. So deklariert dann
%
\begin{lstlisting}[
basicstyle=\scriptsize,numbers=left, numberstyle=\tiny, stepnumber=2, numbersep=5pt]
    Queue<double, 100> db;
\end{lstlisting}
eine Instanz der Klasse \verb|Queue<double, 100>|, welche mit \verb|double|-Zahlen
arbeitet und intern, z.B., ein temp.\  Feld \verb|double tmp[100]| verwaltet.

Das Template-Argument kann mit einem Default-Parameter definiert werden,
%
\begin{lstlisting}[
basicstyle=\scriptsize,numbers=left, numberstyle=\tiny, stepnumber=2, numbersep=5pt]
    template<class T, int n=255>
    class Queue {...};
\end{lstlisting}
wodurch \verb| Queue<double> db; |  ein temp.\  Feld der L"ange~$255$ intern verwalten
w"urde.

Gleitkommazahlen sind nicht als Template-Argumente erlaubt, wohl aber Referenzen auf
Gleitkommazahlen.
%
%
\subsection{Umwandlung einer Klasse in eine Template-Klasse}
\label{p:10.2.3}
%
Ein kleiner Leitfaden um aus einer Klasse (\verb|Komplex|) eine Template-Klasse (\verb|Komplex<T>|)
zu generieren:
\begin{enumerate}
    \item \verb|template <class T>| vor die Klassendeklaration schreiben. \hfill  (\emph{*.h})
    \item Den zu verallgemeinernden Datentyp durch den Template-Parmeter \verb|T| ersetzen.
      \hfill  (\emph{*.h, *.tpp}) \\
        Achtung, vielleicht wollen Sie nicht an allen Stellen diesen Datentyp ersetzen.
    \item Ersetze in allen Parameterlisten, Returntypen und Variablendeklarationen den Klasstentyp \verb|myclass|
      durch die Template-Klasse \verb|myclass<T>|. \hfill  (\emph{*.h, *.tpp})
    \item Vor jede Methodenimplementierung \verb|template <class T>| schreiben. \hfill(\emph{*.tpp}) \\
      Desgleichen vor Funktionen, welche die Klasse \verb|myclass::| als Parameter benutzen. \hfill(\emph{*.tpp}) \\
      Bei \verb|friend|-Funktionen muß man \verb|template <class T>| auch im Deklarationsteil vor der Funktion angeben.
      \hfill  (\emph{*.h})
    \item In der Methodenimplementierung \verb|myclass::| durch \verb|myclass<T>::| ersetzen. \hfill(\emph{*.tpp})
    \item  Im Headerfile das Sourcefile includieren, also \verb|#include "myclass.tpp"|, oder
      gleich alles in das Headerfile schreiben (nicht empfohlen). \hfill  (\emph{*.h})
\end{enumerate}

\subsection{Template-Klasse und \texttt{friend}-Funktionen}
\label{p:10.2.4}
%
Wenn wir den Ausgabeoperator \verb|operator<<| der Template-Klasse \verb|Komplex<T>| als \texttt{friend}-Funktion deklarieren wollen,
dann ändert sich nichts im Implementierungsteil (außer, daß wir direkt auf die Member zugreifen können).
Im Deklarationsteil unserer Klasse müssen wir für diese \verb|friend|-Funktion einen anderen Template-Parameter 
(hier \verb|S|) benutzen.
\begin{lstlisting}[basicstyle=\scriptsize,numbers=left, numberstyle=\tiny, stepnumber=2, numbersep=5pt]

template <class T>
class Komplex
{
  public:
    ...
    Komplex<T> operator+(const Komplex<T>& rhs ) const;
    ...
    template <class S>              // template parameter for following deklaration
    friend ostream& operator<<(ostream& s, const Komplex<S>& rhs);
    ...
};
\end{lstlisting}
%
Diese Art der Deklaration funktioniert ohne Einschränkungen.

\section{Einschränkung der Datentypen bei Templates}
\label{p:10.3}
Die Codes in \S\ref{p:10.1}-\ref{p:10.2} akzeptieren vorerst jeden Templateparameter~\verb|T|.
Eine Deklaration \verb| Komplex<string> as("real","imag");| mit \verb|T = string| ist zwar möglich und selbst die Addition würde 
in Listing~\ref{lst:komplex.tpp} funktionieren, jedoch stellt man sich unter einer komplexen Zahl etwas anderes vor.
Nebenbei, beim Versuch einer Multiplikation würde der Compiler mit einer recht kryptischen Fehlermeldung reagieren.

Es gibt zwei grundsätzliche Möglichkeiten die Anforderungen an den Templateparameter beim 
Kompilieren frühzeitig zu überprüfen:
\begin{itemize}
 \item mit Type Traits und \texttt{static\_assert} [ab C++11],
 \item mit Concepts [ab C++20].
\end{itemize}
Im folgenden werden diese Möglichkeiten für eine Klasse \verb|Komplex<T>| mit 
der Einschränkung, daß \texttt{T} eine Gleitkommazahl sein muß demonstriert.


\subsection{Überprüfung des Templatedatentyps mit Type Traits}
\label{p:10.3.1}
%
Zur Laufzeit besteht die Möglichkeit Datentypen miteinander zu vergleichen bzw. diese 
auf bestimmte Eigenschaften zu testen. 
Diese \ghref{https://en.cppreference.com/w/cpp/header/type_traits}{\texttt{type\_{traits}}} 
enthalten den für uns nützlichen 
\index{type\_trait}
\ghref{https://en.cppreference.com/w/cpp/types/is_floating_point}{Test auf eine Gleitkommazahl} 
welcher in Zeile 8 des Listing~\ref{lst:komplex_17.h} benutzt wird. 
\index{is\_floating\_point}
Das Testergebnis (true oder false) könnte man zur Laufzeit in allen Methoden/Funktionen auswerten.
%
\includecode[linerange={2-2,5-6,14-20,75-77,102-105},label=lst:komplex_17.h]{v_8c_cpp17/komplex.h}{Templateklasse (teilw.) mit Type Traits}
%
Es aber komfortabler wenn bereits der Compiler die Übersetzung des Codes abbricht sobald 
der Templateparameter die gewünschte Bedingung nicht erfüllt. 
Dazu wird im Listing \verb|static_assert| benutzt, der zweite Parameter ist der (informative!) Fehlerstring 
welchen der Compiler im Fehlerfalle ausgibt.

Eine Deklaration von \verb|Komplex<string> as("real","imag");| oder \verb|Komplex<int> ai(1,2);|
hat einen Abbruch der Compilation zur Folge, d.h., die Methoden werden natürlich auch nicht generiert.
Funktionen welche den Templateparameter~\verb|T| ohne die Klasse benutzen, müßten bei Bedarf eine separate Überprüfung wie im Zeile 8 enthalten.

\subsection{Überprüfung des Templatedatentyps mit Concepts}
\label{p:10.3.1}
Seit C++20 kann man über \ghref{https://en.cppreference.com/w/cpp/language/constraints}{Concepts und constraints} 
Anforderungen an den Templateparameter zu stellen. 
Listing~\ref{lst:komplex_20.h} demonstriert dies für unsere Templateklasse und Listing~\ref{lst:komplex_20.tcc} 
für die enstprechende Implementierung der Methoden und Funktionen. In jedem Falle wird ein Kompilierfehler ausgegeben, falls der Templateparameter \verb|T| keine Gleitkommazahl ist.
%
\includecode[linerange={3-3,13-17,79-79,104-107},label=lst:komplex_20.h]{v_8c_cpp20/komplex.h}{Templateklasse (teilw.) unter Nutzung von Concepts}
%
\includecode[linerange={12-18},label=lst:komplex_20.tcc]{v_8c_cpp20/komplex.tcc}{Templatemethoden unter Nutzung von Concepts}
%
Neben den in \verb|<concepts>| \ghref{https://en.cppreference.com/w/cpp/concepts}{vordefinierten} Concepts wie dem hier benutzen 
\ghref{https://en.cppreference.com/w/cpp/concepts/floating_point}{\texttt{floating\_point}} kann man auch eigene Concepts erstellen, siehe dazu~\cite[p.10 and \S4.1]{Grimm:2021:CGD}. 
