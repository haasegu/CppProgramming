\chapter{Einführung in die STL}
\label{p:11}
%
%Container -- Algorithmen -- Iteratoren
%\\
%Nutzung; Nutzung mit eigenen Klassen

%% komplett umschreiben mit Bsp. v_9a
% Container --> Iterator --> Algorithmus
% -  max_element
% -  sort
% -  find
% -  find_if
% Entferne Elemen aus Vektor

\section{Was ist neu?}
\label{p:11.1}

Die \textbf{S}tandard  \textbf{T}emplate \textbf{L}ibrary (STL) enthält
Schablonen (engl.: templates) zur Datenspeicherung, Schablonen für
oft gebräuchliche Algorithmen sowie verallgemeinerte Pointer (Iteratoren) um
beide miteinander zu verbinden.

Die Standardbibliothek enh"alt in der STL (Standard Template Library)
schon viele brauchbare Klassen
und Methoden, welche man nicht immer wieder neu implementieren mu"s.
Wir gehen nur auf einige wenige M"oglichkeiten der STL ein,
es sei auf \cite[\S49]{Meyers:1998:ECP},  \cite[\S6.4.1]{Meyers:1997:MEC},\cite[\S10]{Yang:2001:COO},
\cite[567-711]{KirchPrinz:2002:OOP}, \cite{KuhlinsSchader:2002:DCS} verwiesen.
Die Klassen und Methoden der Standardbibliothek sind im Namensraum \texttt{std} untergebracht,
soda"s, z.B., \texttt{cout}
via den Namespace \texttt{std::cout} aufgerufen werden mu"s (nicht zu Verwechseln mit
dem Scope-Operator) oder man gibt einzelne Komponenten des Namensraumes "uber
\verb|using std::cout| bzw. den gesamten Namensraum "uber
\verb|using namespace| frei.
%\exfile{A1/a1\_1.cpp} frei.

Die STL basiert auf drei grunds"atzlichen Konzepten:
Containern, Iteratoren und Algorithmen.
\index{Container}\index{Iteratoren}\index{Algorithmen}
Die Container beinhalten (mehrere) Objekte deren Templates konkretisiert werden,
und in welchen mit Hilfe von Iteratoren navigiert werden kann.
In Fortf"uhrung dessen sind Algorithmen
Funktionen welche auf Containern (mittels der Iteratoren) bestimmte
Operationen durchf"uhren.

\begin{lstlisting}[caption={Intro STL},label=lst:stl_a,
basicstyle=\scriptsize,numbers=left, numberstyle=\tiny, stepnumber=2, numbersep=5pt]

#include <vector>                       //  vector<T>
#include <algorithm>                    //  find()
using namespace std;
...
{
    vector<int> v(10);                  // container
    // Init v
    ....
    vector<int>::iterator it;           // iterator
    it = find(v.begin(), v.end(), 4);   // algorithm, iterators
}
\end{lstlisting}
Das kurzes Codefragment~\ref{lst:stl_a} enthält alle drei Konzept der STL:
\begin{itemize}
    \item \ghref{http://www.cplusplus.com/reference/stl/}{\emph{Container}}: unterschiedliche Datenstrukturen zur Speicherung von Elementen, z.B.,
      \verb|vector|, \verb|list|, \verb|stack|.
      \\
      Container sind eigenen Templateklassen mit eigenen Methoden (z.B., \verb|erase|, \verb|assign|)
      und zugehörigen Iteratoren.
    \item \ghref{http://www.cplusplus.com/reference/algorithm/}{\emph{Algorithmen}}:
      unabhängig vom konkreten Containers implementierte Algorithmen, z.B.,
      \verb|find|, \verb|find_if|, \verb|sort|, \verb|unique|, \verb|count|, \verb|accumulate|.{}
      \\[0.5ex]
      $\Longrightarrow$ Algorithmen und Container haben nichts miteinander zu tun.
    \item \ghref{http://www.cplusplus.com/reference/iterator/}{\emph{Iteratoren}}:
       sind das Bindeglied zwischen Containern und Algorithmen.
       Ein klassischer C-Pointer \verb|int*| kann als Iterator betrachtet und auch als solcher benutzt werden.
\end{itemize}
In obigem Code~\ref{lst:stl_a} sind \verb|v.begin()| und \verb|v.end()| Methoden des Containers
\verb|vector<int>| welche Iteratoren auf das erste und das hinterletzte Element zurückliefern,
also den Definitionsbereich für den Algorithmus~\verb|find|.
Der Algorithmus liefert wiederum einen Iterator auf das gefundenen Element zurück welcher
in \verb|it| gespeichert wird. Mit der Dereferenzierung \verb|*it| können wir direkt auf
den Containereintrag zugreifen.
% allerdings nur dann wenn \verb| it != v.end()| ist.



Das Finden des größten Elements eines Vektors, für welches wir in Listing~\ref{lst:fkt_template_s}
auf Seite~\ref{lst:fkt_template_s} eine eigene Templatefunktion geschrieben haben, läßt sich mittels
der STL ganz einfach lösen.
\begin{lstlisting}[caption={Größter Vektoreintrag: STL anwenden.},label=lst:stl_b,
basicstyle=\scriptsize,numbers=left, numberstyle=\tiny, stepnumber=2, numbersep=5pt]
#include <iostream>
#include <vector>
#include <algorithm>
using namespace std;

int main()
{
 const int N = 10;
 vector<int>    ia(N);
 vector<double> da(N);
          // Vektoren initialisieren
 ...

 vector<int>::iterator pm;
 pm = max_element(ia.begin(),ia.end());
 int im = *pm;
 cout << "   int-Feld(max):  " << im << endl;

 double dm = *max_element(da.begin(),da.end());
 cout << "double-Feld(max):  " << dm << endl;

 return 0;
}
\end{lstlisting}
\begin{itemize}
    \item Zeile 14 definiert den notwendigen Iterator.
    \item Zeile 15 bestimmt den Iterator des maximalen Elements des gesamten Vektors.
    \item Zeile 16 dereferenziert den Iterator und speichert den Wert.
    \item Zeile 19 kombiniert Zeilen 14-16 in einem Schritt für den entsprechenden Container.
    \item Der Algorithmus~\verb|max_element| benötigt den Vergleichsoperator \verb|operator<|
    für die Elemente des Containers.
\end{itemize}

Einige allgemeine Hinweise zur Benutzung der STL-Algorithmen:
\begin{enumerate}
    \item Jeder STL-Algorithmus benötigt einen, von Iteratoren begrenzten, Input-bereich 
      eines Containers auf welchen der Algorithmus anzuwenden ist, z.B., \verb|ia.begin(),ia.end()| in 
      obigem Beispiel.
    \item Manche STL-Algorithmen benötigen zusätzlich einen Output-bereich dessen Container
     \textbf{eine ausreichende Länge} haben muß, z.B., in 
     \ghref{http://www.cplusplus.com/reference/algorithm/copy/}{\texttt{copy}}. 
    \item Viele STL-Algorithmen haben mehrere Aufrufmöglichkeiten, einmal mit einem Standardvergleichoperator 
    und zum zweiten mit einer spezifischen Vergleichsfunktion. 
    Siehe dazu die \ghref{http://www.cplusplus.com/reference/algorithm/max_element}{Info zu \texttt{max\_element}} 
    und das Beispiel dort. Näheres im nächsten Abschnitt.
\end{enumerate} 

\section{Wie benutze ich \texttt{max\_element} ?}
\label{p:11.2}
Wir beschränken uns in diesem Abschnitt auf den Algorithmus \verb|max_element| um einige beachtenswerte Details 
hervorzuheben. 
%
\subsection{Container mit Standarddatentypen}
\label{p:11.2.1}
In Fortsetzung obigen Beispiels betrachten wir den folgenden 
\ghref{imsc.uni-graz.at/haasegu/Lectures/Kurs-C/Beispiele/v_stl_intro.zip}{C++11-Code}
der Anwendung von \verb|max_element| auf
einen Container \verb|vector<int>|.

%
\paragraph{Nutzung des Standardvergleichsoperators} 
\label{p:11.2.1.1}
%
Da die Elemente des Containers vom Typ \verb|int| sind, nutzt die 
Standardversion des Algorithmus den Vergleichsoperator \verb|<| dieser
Klasse \verb|int| in Zeile~10 des Listings~\ref{lst:v_stl_int_a}.
%
\includecode[linerange={3-6,35-37,40-43,62-64},label=lst:v_stl_int_a]{v_stl_intro/main.cpp}
{Algorithmus mit Standardvergleichoperator \texttt{operator<}}
%
Die erweiterte Version des Algorithmus erfordert eine boolesche Vergleichsfunktion welche auf 
drei verschiedene Weisen in den nächsten Beispielen definiert wird.
%
\paragraph{Nutzung einer Vergleichsfunktion} 
\label{p:11.2.1.2}
%
In den Zeilen 6-9 wird eine boolesche Vergleichsfunktion~\texttt{fkt\_abs\_less} definiert, welche in Zeile~15 
zum betragsmäßigen Vergleich der Elemente benutzt wird. 
%
\includecode[linerange={3-6,20-24,35-37,44-44,48-50,62-64},label=lst:v_stl_int_b]{v_stl_intro/main.cpp}
{Algorithmus mit Vergleichsfunktion}
%

\paragraph{Nutzung eines Funktors} 
\label{p:11.2.1.3}
%
In den Zeilen 6-12 wird eine Funktorklasse (-struktur) \texttt{funktor\_abs\_less} 
deklariert, welche nur den Klammeroperator \verb|operator()| als Methode mit booleschem Rückgabewert besitzt.
Dieser Klammeroperator wird beim Vergleich im Algorithmus in Zeile~18 benutzt.
%
\includecode[linerange={3-6,26-33,35-37,44-44,52-54,62-64},label=lst:v_stl_int_c]{v_stl_intro/main.cpp}
{Algorithmus mit Funktor}
%

\paragraph{Nutzung einer Lambda-Funktion} 
\label{p:11.2.1.4}
%
Mit C++11 kann die Definition der kurzen Funktion~\texttt{fkt\_abs\_less} aus~\S\ref{p:11.2.1.2} 
gleich in die Rufzeile des Algorithmus geschrieben werden.
Diese Lambda-Funktion\index{Lambda-Funktion} wird in Zeile~10 definiert und gleichzeitig angewendet.
Mehr zu Lambda-Funktionen in Alex Allains sehr gutem 
\ghref{http://www.cprogramming.com/c++11/c++11-lambda-closures.html}{Tutorial}.
%
\includecode[linerange={3-6, 35-37,44-44,57-60,62-64},label=lst:v_stl_int_d]{v_stl_intro/main.cpp}
{Algorithmus mit Lambda-Funktion}
%


\subsection{Container mit Elementen der eigenen Klasse}
\label{p:11.2.2}
Wenn wir statt der Standarddatentypen die eigene Klasse \verb|komplex| aus \S\ref{p:9.1} benutzen, 
dann muß unsere Klasse einige Anforderungen erfüllen. 
Insbesondere müssen zumindest die Vergleichsoperatoren \verb|operator<| und \verb|operator==| vorhanden sein. 
Desgleichen ist der Zuweisungsoperator  \verb|operator=| für Kopieroperationen nötig 
(es reicht normalerweise der vom Compiler generierte Zuweisungsoperator).
%
%
\paragraph{Nutzung des Standardvergleichsoperators} 
\label{p:11.2.2.1}
\mbox{}
\includecode[linerange={3-7,36-38,42-43,63-64},label=lst:v_stl2_int_a]{v_stl_intro2/main.cpp}
{Algorithmus mit Standardvergleichoperator \texttt{operator<}}
Der Standardvergleichsoperator muß für unsere Klasse deklariert und definiert sein:
\includecode[linerange={1-1,47-50},label=lst:v_stl2_int_b]{v_stl_intro2/komplex.cpp}
{Standardvergleichoperator \texttt{operator<} für \texttt{Komplex}}

\paragraph{Nutzung einer Vergleichsfunktion} 
\label{p:11.2.2.2}
%
In den Zeilen 7-10 wird eine boolesche Vergleichsfunktion~\texttt{fkt\_abs\_less} definiert welche in Zeile~14 
zum betragsmäßigen Vergleich der Elemente benutzt wird. 
%
\includecode[linerange={3-7,21-25,36-38,50-50,63-64},label=lst:v_stl2_int_c]{v_stl_intro2/main.cpp}
{Algorithmus mit Vergleichsfunktion}
%
Hierzu benötigen wir noch die Funktion \verb|abs| für unsere Klasse.
\includecode[linerange={118-121},label=lst:v_stl2_int_c_abs]{v_stl_intro2/komplex.h}
{Inline-Funktion \texttt{abs}}
Natürlich hätten wird \verb|abs| auch als Methode der Klasse \texttt{Komplex} implementieren können, dann 
müßte man nur die Vergleichsfunktion etwas umschreiben (\verb|return a.abs()<b.abs();|)
%

Die Verwendung von Funktor bzw.\  Lambda-Funktion erfolgt analog zu \S\ref{p:11.2.1}.

\section{Einige Grundaufgaben und deren Lösung mit der STL}
\label{p:11.3}

Die Grundidee bei allen STL-Algorithmen besteht darin, daß nur Iteratoren auf Container übergeben 
werden. Damit besteht auch für den Algorithmus keinerlei Möglichkeit, die Länge eines Containers 
zu verändern. Dies kann nur über Methoden des Containers erfolgen, z.B., \verb|erase| oder \verb|resize|.{}
Manche Algorithmen erfordern spezielle Eigenschaften der Iteratoren wie wahlfreien Zugriff (engl.: random access) 
welche nicht von allen Containeriteratoren bereitgestellt werden können. Für diese steht dann eine entsprechende Methode 
zur Verfügung, siehe~\S\ref{p:11.3.2}.

Die folgenen Beispiele demonstrieren die Nutzung der STL für einige Grundaufgaben an Hand 
der Container \verb|vector| (random access iterator) und \verb|list| (bidirectional iterator). 
%
\includecode[linerange={3-30,56-58,124-124,186-187},label=lst:v_stl3_main]
{v_stl_intro3/main.cpp}{Rahmencode für STL-Demonstration}
%
Der \emph{range-for}-Loop in Zeilen 20-23 ist die allgemeingültige Lösung um über 
den Gesamtbereich eines beliebigen Containers zu itererieren, d.h., dies wäre auch für die Ausgabe des 
vectors anwendbar. Da der \verb|list|-Iterator nicht wahlfrei ist, kann der for-Loop aus den Zeilen 10-13 
nicht auf diesen angewandt werden. 


\subsection{Kopieren von Daten}
\label{p:11.3.1}
%
\includecode[linerange={64-65},label=lst:v_stl3_copy_vec]
{v_stl_intro3/main.cpp}{Kopieren eines Vektors}
%
%
\includecode[linerange={130-131},label=lst:v_stl3_copy_list]
{v_stl_intro3/main.cpp}{Kopieren einer Liste}
%

\subsection{Aufsteigendes Sortieren von Daten}
\label{p:11.3.2}
%
\includecode[linerange={70-70},label=lst:v_stl3_sort_vec]
{v_stl_intro3/main.cpp}{Aufsteigendes Sortieren eines Vektors}
%
%
\includecode[linerange={137-137},label=lst:v_stl3_sort_list]
{v_stl_intro3/main.cpp}{Aufsteigendes Sortieren einer Liste}
%
\subsection{Mehrfache Elemente entfernen}
\label{p:11.3.3}
Der Algorithmus \verb|unique| entfernt nur diejenigen Elemente aus einem Container, 
welche darin \textbf{benachbart} gespeichert sind. 
Wenn man also aus einem Container alle mehrfachen Elemente, unabhängig von deren originaler Nachbarschaftsbeziehung, 
entfernen will, dann muß der Container vorher sortiert werden wie in \S\ref{p:11.3.2}.
Die Länge des Containers muß anschließend mit der Methode \verb|erase| korrigiert werden.
%
\includecode[linerange={76-79},label=lst:v_stl3_unique_vec]
{v_stl_intro3/main.cpp}{Entfernen mehrfacher Elemente aus einem geordneten Vektor}
%
%
\includecode[linerange={143-145},label=lst:v_stl3_unique_list]
{v_stl_intro3/main.cpp}{Entfernen mehrfacher Elemente aus einer geordneten Liste}

%
%
\subsection{Kopieren von Elementen welche einem Kriterium nicht entsprechen}
\label{p:11.3.4}
Wie beim Kopieren muß der Zielcontainer groß genug sein.
In unsererm Beispiel kopiert der Algorithmus 
%\verb|remove_copy_if| 
\verb|copy_if| 
nur diejenigen Elemente welche nichtnegativ 
(Boolesche Funktion \verb|IsNotNegative|) sind. 
Die Länge des Containers muß anschließend mit der Methode \verb|erase| korrigiert werden.
%
\includecode[linerange={41-41},label=lst:v_stl3_rcif_fkt]
{v_stl_intro3/main.cpp}{Boolesche Funktion}
%
\includecode[linerange={84-87},label=lst:v_stl3_rcif_vec]
{v_stl_intro3/main.cpp}{Kopieren von bestimmten Elementen aus einem Vektor}
%
%
\includecode[linerange={151-153},label=lst:v_stl3_rcif_list]
{v_stl_intro3/main.cpp}{Kopieren von bestimmten Elementen aus einer Liste}
%
%Mit dem Algorithmus \verb|copy_if| hätten wir das Ziel leichter erreicht aber diesen Algorithmus 
%gibt es erst seit C++11. Dazu sagt Bjarne Stroustrup: 
%\emph{Aufgrund eines mir unterlaufenen Fehlers fehlt dieser Algorithmus im ISO-Standard 1998. 
%Der Fehler wurde zwar inzwischen behoben, aber es gibt immer noch Implementierungen ohne \texttt{copy\_if}.} 
%\cite[\S21.7.4]{Stroustrup:2010:EPC}

%
%
\subsection{Absteigendes Sortieren von Elementen}
\label{p:11.3.5}
%
\includecode[linerange={111-111},label=lst:v_stl3_sortb_vec]
{v_stl_intro3/main.cpp}{Absteigendes Sortieren eines Vektors}
%
%
\includecode[linerange={176-176},label=lst:v_stl3_sortb_list]
{v_stl_intro3/main.cpp}{Absteigendes Sortieren einer Liste}
%
Statt des Funktors \verb|greater<int>()| (benutzt intern \verb|int::operator>|) kann 
natürlich auch eine beliebige andere Boolesche Funktion benutzt werden 
(Listings \ref{lst:v_stl_int_b}-\ref{lst:v_stl_int_d}).

%
%
\subsection{Zählen bestimmter Elemente}
\label{p:11.3.6}
%
%
\includecode[linerange={117-117},label=lst:v_stl3_count_vec]
{v_stl_intro3/main.cpp}{Zählen bestimmter Elemente eines Vektors}
%
%
\includecode[linerange={182-182},label=lst:v_stl3_count_list]
{v_stl_intro3/main.cpp}{Zählen bestimmter Elemente einer Liste}
%

%
%
\subsection{Sortieren mit zusätzlichem Permutationsvektor}
\label{p:11.3.7}
% 
Der Sortieralgorithmus der STL liefert keinen Indexvektor der Umordnung zurück.
Folgende zwei Lösungen werden bei 
\ghref{http://stackoverflow.com/questions/1577475/c-sorting-and-keeping-track-of-indexes}{\emph{stackoverflow}} 
vorgeschlagen.
\begin{itemize}
    \item Statt des zu sortierenden \verb|vector<double>| wird 
       \verb|vector< pair<double,int> >| sortiert, wobei der zweite Parameter mit den Indizes $0,1,2,\ldots$ 
       inialisiert wird. 
       Hinterher muß man nur noch die zweiten Parameter in einen Indexvektor extrahieren.
    \item Meine favorisierte Lösung ist die folgende, welche eine Lambda-Funktion nutzt und 
    statt den Datenvektor umzuordnen nur den Permutationsvektor der Umordnung zurückliefert. 
    Auf das Minimum des Datenvektors kann anschließend mittels \verb| v[idx[0]] | zugegriffen werden (\htmladdnormallinkfoot{Code}{http://imsc.uni-graz.at/haasegu/Lectures/Kurs-C/Beispiele/sort_index.cpp}).
\begin{lstlisting}[caption={Sortieren mit Permutationsvektor},label=lst:sort_permutvektor,
basicstyle=\scriptsize,numbers=left, numberstyle=\tiny, stepnumber=2, numbersep=5pt]
template <typename T>
vector<size_t> sort_indexes(const vector<T> &v) 
{
  // initialize original index locations
  vector<size_t> idx(v.size());
  for (size_t i = 0; i != idx.size(); ++i) idx[i] = i;

  // sort indexes based on comparing values in v
  sort(idx.begin(), idx.end(),
       [&v](size_t i1, size_t i2) {return v[i1] < v[i2];});

  return idx;
}

//   Application
vector<double> v = {....};        // initialize v
vector<size_t>  idx =  sort_indexes(v);
\end{lstlisting}    
\end{itemize}


%
\subsection{Ausgabe der Elemente eines Containers}
\label{p:11.3.8}
% 
Wir nehmen als Beispielcontainer wieder \texttt{std::vector<T>}, die Codes sind leicht auf 
andere Container übertragbar. 
Abgesehen von der recht plumpen Lösung über eine Funktion \verb|void print(vector<T> const & x)| 
ist das Definieren eines Ausgabeoperators die beste Lösung.
%
\begin{lstlisting}[caption={Vektorausgabe: Version 1},label=lst:cout_vector1,
basicstyle=\scriptsize,numbers=left, numberstyle=\tiny, stepnumber=2, numbersep=5pt]
#include <iostream>
#include <vector>

template <class T>
ostream& operator<<(ostream &s, vector<T> const &x)
{
   for (const auto &pi: x)
   {
      s << pi << "  ";
   }
   return s;
}

//   Application
vector<double> v{....};        // initialize v
cout << v << endl;
\end{lstlisting}  
%
Der range-for-Loop in obigem Listing ersetzt den Loop von \verb|cbegin(x)| bis \verb|cend(x)| 
und kann auch durch ein direktes Kopieren auf den Ausgabestrom ersetzt werden.
%
\begin{lstlisting}[caption={Vektorausgabe: Version 2},label=lst:cout_vector1,
basicstyle=\scriptsize,numbers=left, numberstyle=\tiny, stepnumber=2, numbersep=5pt]
#include <iostream>
#include <iterator>
#include <vector>

template <class T>
ostream& operator<<(ostream &s, vector<T> const &x)
{
  copy(cbegin(x), cend(x), ostream_iterator<T>(s, "  "));
  return s;
}

...
//   Application
vector<double> v{....};        // initialize v
cout << v << endl;
...
\end{lstlisting}  
%
%
%
\section{Allgemeine Bemerkungen zur STL}
\label{p:11.4}
%
Die wichtigsten Quelle zur STL sind \ghref{https://en.cppreference.com/w/}{\texttt{cppreference.com}}  oder  \ghref{http://www.cplusplus.com}{\texttt{cplusplus.com}}, 
da man hier schnell Details zu einzelnen Algorithmen, zu Iteratoren oder Methoden von Containern findet. 

Die meisten STL-Algorithmen existieren in einer Standardvariante, in welcher ein Operator 
der Containerelemente benutzt wird (\verb|operator<| in \verb|sort|) und einer weiteren Variante, in welcher 
ein entsprechender Operator (oder eine Funktion) übergeben wird, meist über eine Lambda-Funktion. 
Genauso gibt es häufig Zwillings-Algorithmen wie \verb|count| und \verb|count_if|. 
%
%
%
\section{\mbox{}$^{*}$Parallelität in der STL [C++17]}
\label{p:11.5}
Ab C++-17 sind zu vielen Algorithmen auch parallele Versionen verfügbar 
welche als zusätzlichen ersten Parameter eine \emph{execution policy} entgegennehmen,
siehe dazu Versionen~(2) und~(4) von \ghref{https://en.cppreference.com/w/cpp/algorithm/sort}{\texttt{sort}}
sowie \ghref{https://en.cppreference.com/w/cpp/algorithm/execution_policy_tag_t}{\texttt{std::execution::par}}.
Dabei werden alle verfügbaren Cores der CPU(s) benutzt, 
dank Hyperthreading werde normalerweise doppelt so viele Threads für die Parallelisierung benutzt. 
Diese Anzahl der benutzen Threads ist nicht steuerbar, im Gegensatz zu einer OpenMP-Parallelisierung.
\\
\underline{Achtung:} Beim Linken muß das Flag \texttt{-ltbb} nach den Objektfiles hinzugefügt werden.\\
Gegebenenfalls muß die Option \texttt{-std=c++17} beim Compilieren angegeben werden.

Das Sortieren eines Containers läßt sich wie folgt beschleunigen. 
%
\begin{lstlisting}[caption={Paralleles Sortieren},label=lst:par_sort,
basicstyle=\scriptsize,numbers=left, numberstyle=\tiny, stepnumber=2, numbersep=5pt]
// thread_17
#include <algorithm>
#include <execution>          // execution policy
#include <vector>

...
{
    size_t const N = 1<<25;
    vector<double> v(N);
    iota(v.begin(), v.end(), 1);
    std::shuffle(v.begin(), v.end(), std::mt19937{std::random_device{}()});

    sort(std::execution::par, v.begin(), v.end());
}
...
\end{lstlisting}
%
In obigem Beispiel \ghref{https://imsc.uni-graz.at/haasegu/Lectures/Kurs-C/SS21/thread_17.zip}{\texttt{thread\_17}} 
und einem Vector mit 33.554.432 Elementen
beträgt die erzielte Beschleunigung (SpeedUp) gegenüber der sequentiellen Variante 7.5 für eine 8-core CPU (AMD Ryzen 7 3800X) 
bzw.\  16.5 für eine 32-core CPU (AMD EPYC 7551P). 
\\
Die erzielten Beschleunigungen ändern sich mit der Anzahl der Elemente 
und mit deren Datentyp (Speicherbandbreite; memory bandwidth). 
%

%%\subsection{Nutzung einer Klassenhierarchie in der STL}
%%\label{sec:A10.3}
%%%
%%\subsubsection{Die Klassenhierarchie}
%%\label{sec:A10.3.1}
%%%
%%\subsubsection{Sortieren der Instanzen einer Klasse}
%%\label{sec:A10.3.2}
%%%
%%\subsubsection{Sortieren der Instanzen aus der Klassenhierarchie}
%%\label{sec:A10.3.3}
%%
