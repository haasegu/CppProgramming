\chapter{Variablen und Datentypen}
\label{p:2}
%
\section{Variablen}
\label{p:2.1}
Jedes sinnvolle Programm bearbeitet Daten in irgendeiner Form.
Die Speicherung dieser Daten erfolgt in Variablen.\index{Variable|(}

Allgemeine Form der Variablenvereinbarung: \\
\centerline {\texttt {
  <typ> <bezeichner1> [, bezeichner2] ;
}}

\noindent
Die Variable
\begin{minipage}[t] {0.8\textwidth}
 \begin{enumerate}
   \renewcommand {\labelenumi}{\roman{enumi})}
   \item ist eine symbolische Repr"asentation (Bezeichner/Name) f"ur
   	den Speicherplatz von Daten.
   \item  wird beschrieben durch Typ und
   	Speicherklasse.\index{Variable!Typ}\index{Variable!Speicherklasse}.
   	Der Datentyp \texttt{<typ>} einer Variablen bestimmt die zulässigen Werte
   dieser Variablen.
   \item Die Inhalte der Variablen, d.h., die im Speicherplatz befindlichen
   	Daten, "andern sich w"ahrend des Programmablaufes.
 \end{enumerate}
\end{minipage}

%
\subsection{Einfache Datentypen}
\label{p:2.2}
%
Daten im Computer basieren auf dem binären Zahlensystem von
Gottfried Wilhelm Leibniz\footnote{Gottfried Wilhelm Leibniz,
\textborn 1.7.1646 in Leipzig, \textdied 14.11.1716 in Hannover}
worin alle ganzen Zahlen durch die Summation von
mit~0 oder~1 gewichteten Zweierpotenzen dargestellt werden können. So wird die Zahl
117 im Dezimalsystem durch
$$ 117_{(10)} =
01110101_{(2)} = 0 \cdot 2^{7} + 1 \cdot 2^{6} + 1 \cdot 2^{5} +
 1 \cdot 2^{4} + 0 \cdot 2^{3} + 1 \cdot 2^{2} + 0 \cdot 2^{1} + 1 \cdot 2^{0}$$
binär dargestellt.
Die grundlegende Informationseinheit im Computer ist das \emph{Bit} welches
genau einen der zwei Zustände~1 oder~0 annehmen kann
(ja/nein, true/false). Somit werden in obiger Binärendarstellung von 117
genau 8~Bit benötigt. Diese 8~Bit stellen die kleinste Grundeinheit des
Datenzugriffs bei heutigen Computern dar und werden als ein \emph{Byte} bezeichnet.
\index{bit}\index{Byte}

%
Aufbauend auf der Speichereinheit Byte sind die in Tabelle~\ref{tab:2.1.1}
dargestellten, \ghref{https://en.cppreference.com/w/cpp/language/types}{grundlegenden Datentypen} in C++ vorhanden.
%
\begin{table}[htb]
	\label{tab:2.1.1}
\mbox{}\hfill
\begin{tabular}{l@{\quad}c@{\quad}l@{\quad}p{0.3\textwidth}}
    \hline\hline
 Typ		& \hspace{-6ex}Speicherbedarf& Inhalt	& mögliche Werte \\
 		& \hspace{-6ex}in Byte	&		&	\\  \hline\hline
 \texttt{byte}	& 1		& Byte & $0x00000000 \ldots 0x11111111$
	\\ \hline 
 \texttt{char}	& 1		& ASCII-Zeichen & 'H', 'e', '$\backslash$n'
	\\
 \texttt{wchar\_t}	& 2$|$4 & Unicode (Win$|$Unix) & Ä. ß, $\gamma$, $\Omega$, $\daleth$, $\aleph$
	\\    
     \hline
 \texttt{bool}	& 1		& Booleanvariable & \verb|false|, \verb|true|\quad [erst ab C90]
	\\ \hline
%
 \texttt{signed char} & 1		&	&  $[-128, 127]$ ; -117, 67 \\
 \texttt{unsigned char} & 1		&	&    $[0, 255]$ ;  139, 67\\
 \texttt{short} [\texttt{int}] 	& 2		& 	& $[-2^{15},2^{15}-1]$ ; -32767 \\
 \texttt{unsigned short} [\texttt{int}] & 2		& 	& $[0,2^{16}-1]$ ;  40000\\
 \texttt{int}	& 4		&	Ganze Zahlen	&  $[-2^{31},2^{31}-1]$ \\
 \texttt{unsigned} [\texttt{int}]	& 4		&	(integer)	&  $[0,2^{32}-1]$ \\
 \texttt{long}  [\texttt{int}]  	& 4		&	& wie \texttt{int} \\
 \texttt{unsigned long}  [\texttt{int}] 	& 4		&		& wie \texttt{unsigned int} \\
 \texttt{long long}  [\texttt{int}]  		& 8		&		&  $[-2^{63},2^{63}-1]$  \\
 \texttt{unsigned long long}  [\texttt{int}]& 8		&		& $[0,2^{64}-1]$ \\
 \texttt{size\_t}  & 		&	implementierungsabhängig	&  \texttt{unsigned long long int}
 \\ \hline
%
 \texttt{float} & 4		& Gleitkommazahlen& 1.1, -1.56e-32 \\
 \texttt{double}& 8		& (floating point numbers)		& 1.1, -1.56e-132, 5.68e+287
 	\\ \hline\hline
\end{tabular}
\hfill\mbox{}
	\caption{Speicherbedarf grundlegender Datentypen [Stand 03/2019]}
\end{table}
\index{char}\index{bool}\index{int}\index{float}\index{double}
%
\begin{itemize}
 \item Standardcharacterdaten speichern in einem Byte (=~8~bit) genau ein
  \ghref{http://de.wikipedia.org/wiki/Ascii\#ASCII-Tabelle}{ASCII}- oder Sonderzeichen, d.h.,
  die kodierten Buchstaben, Ziffern und Sonderzeichen werden durch eine ganze Zahl aus $[0,255=2^{8}-1]$
  (\texttt{unsigned char}) dargestellt. 
 \item Der Speicherbedarf von ganzzahligen Datentypen (\texttt{short int}, \texttt{int},
 \texttt{long int}, \texttt{long long int} )\index{Integer}
 	kann von Compiler und Betriebssystem (16/32/64 bit) abh"angen.
	Es empfiehlt sich daher, die einschl"agigen Compilerhinweise zu lesen
	bzw.\  mit dem \texttt{sizeof}-Operator mittels
	\texttt{sizeof(<typ>)} oder \texttt{sizeof(<variable>)}\index{sizeof()}
	die tats"achliche Anzahl der ben"otigten Bytes zu ermitteln.
	Siehe dazu auch das Listing~\ref{lst:DataTypes.cpp}.
 \item Wir werden meist den Grundtyp \texttt{int} f"ur
 	den entsprechenden Teilbereich der ganze Zahlen und
 	\texttt{unsigned int} f"ur nat"urliche Zahlen verwenden.
	Die Kennzeichnung \texttt{unsigned} sowie das ungebräuchliche \texttt{signed} kann auch in
	Verbindung mit anderen Integertypen verwendet werden.
    Falls der gewählte Datentyp stets genau 16 bit beinhalten soll, dann können 
    \ghref{https://en.cppreference.com/w/cpp/types/integer}{Fixed Integer Types} wie \verb|int16_t| benutzt werden.
 \item Floating point numbers sind im Standard \ghref{https://de.wikipedia.org/wiki/IEEE_754}{IEEE 754} definiert.
 Die Anzahl der Mantissenbits (\texttt{float}: 23; \texttt{double}: 52) bestimmt deren relative Genauigkeit,
  d.h., das Maschinen-$\varepsilon$, welches die kleinste Zahl des Datentyps darstellt
  für die $1+\varepsilon > 1$ noch gilt
  (\texttt{float}: $2^{-(23+1)}$; \texttt{double}: $2^{-(52+1)}$).
  Siehe dazu auch das \ghref{https://de.wikipedia.org/wiki/Gleitkommazahl\#Hidden_bit}{hidden bit}.
\end{itemize}
\includecode[linerange={4-8,15-15,65-66,75-76}]{DataTypes.cpp}{Abfrage der Speichergröße von Datentypen}
%
%
%
\subsection{Bezeichner von Variablen}
\label{p:2.1.2}
%
Die Variablenbezeichner müssen gewissen Regeln folgen (\ghref{http://de.wikipedia.org/wiki/C_(Programmiersprache)\#Deklarationen}{wiki}):
\begin{itemize}
	\item Das erste Zeichen eines Bezeichners einer Variablen muss ein Buchstabe oder Unterstrich sein.
	\item Die folgenden Zeichen dürfen nur die Buchstaben A–Z und a–z, Ziffern und der Unterstrich sein.
	\item Ein Bezeichner darf kein Schlüsselwort der Sprache – zum Beispiel if, void und auto – sein.
 \item C/C++ unterscheidet zwischen Gro\ss - und Kleinschreibung, d.h.,
 	\texttt{ToteHosen} und \texttt{toteHosen} sind unterschiedliche
	Bezeichner!
	\bspfile{Ex210.cpp}
 %\item Laut originalem C-Standard sind die ersten 8 Zeichen eines
 	%Variablenbezeichners
 	%signifikant, d.h., \texttt{a2345678A} und \texttt{a2345678B}
	%w"urden nicht mehr als verschiedene Bezeichner wahrgenommen.
	%Mittlerweile sehen Compiler mehr Zeichen als signifikant an
	%(C9X Standard: 63 Zeichen).
\end{itemize}
%
\begin{table}[hbt]
	\label{tab:2.1.2}
\mbox{}\hfill
\begin{tabular}{l@{$\;\;$}||@{$\;\;$}r@{\qquad\qquad}p{0.3\textwidth}}
  G"ultig & Ung"ultig & Grund \\ \hline
  \texttt{i} \\
  \texttt{j} \\
  \texttt{ijl} \\
  \texttt{i3}	& \texttt{3i}	& 3 ist kein Buchstabe \\
  \texttt{\_3a}	& \texttt{\_3*a} & * ist Operatorzeichen \\
  \texttt{Drei\_mal\_a} & \texttt{b-a} & - ist Operatorzeichen \\
  \texttt{auto1} & \texttt{auto} & \texttt{auto} ist Schl"usselwort
\end{tabular}
\hfill\mbox{}
\caption{Einge erlaubte und nicht erlaubte Variablenbezeichner}
\end{table}
%
\index{Variable|)}
%
%
%
\subsection{G"ultigkeit von Bezeichnern}
\label{p:2.1.3}
%
Die Bezeichner von Variablen, Konstanten etc. sind im Regelfall
lokal, d.h., ein Bezeichner ist nur innerhalb seines, von \verb|{ }|
begrenzten Blockes von Codezeilen gültig, in welchem dieser Bezeichner
deklariert wurde. Daher nennt man diesen Block den
Gültigkeitsbereich (scope)\index{Block}\index{scope}
dieser Variablen.
Siehe dazu Listing~\ref{lst:HelloWorld.cpp} und ~\S\ref{p:4.2}.
%
%
\subsection{Konstante mit Variablennamen}
\label{p:2.2.6}
%
\index{Konstante|(}%
Wird eine Variablenvereinbarung zus"atzlich mit dem
Schl"usselwort \texttt{const} gekennzeichnet, so kann diese
Variable nur im Vereinbarungsteil initialisiert werden und danach nie
wieder, d.h., sie wirkt als eine Konstante.
\includecode[firstline=6]{Ex226.cpp}{Definition und Deklaration von Konstanten und Variablen}
%
Nach der Deklaration in Zeile~6 ist die Variable~\texttt{i} noch nicht mit einem Wert belegt,
d.h., sie hat einen undefinierten Status.
In Zeile~7 würde der Wert von~\texttt{i} zufällig sein, je nachdem was gerade
in den reservierten 4~Byte als Bitmuster im Computerspeicher steht.
Damit wäre das Ergebnis einer Berechung mit~\texttt{i} nicht vorhersagbar
(nicht deterministisch) und damit wertlos.

Also sollte~\texttt{i} gleich bei der Deklaration auch definiert (initialisiert) werden.
In C++ kann man ausnutzen, daß Deklarationsteil und Implementierungsteil gemischt werden können.
Dies wird im Listing~\ref{lst:Ex226_b.cpp} ausgenutzt, sodaß die Variable~\texttt{i}
erst in Zeile~4 deklariert wird, wo dieser auch gleich ein sinnvoller
Wert zugewiesen werden kann.
%
\includecode[linerange={11-13,15-16}]{Ex226_b.cpp}{Variablen erst bei Gebrauch deklarieren}
\index{Konstante|)}
%
%
\section{Literale Konstanten}
\label{p:2.22}
%
Die meisten Programme, auch \textit{HelloWorld.cpp}, verwenden
im Programmverlauf unver"anderliche Werte, sogenannte Literale und Konstanten.
\index{Literale|(}
Literale Konstanten sind Werte ohne Variablenbezug welche in Ausdrücken explizit angegeben werden und zusätzliche
Typinformationen enthalten können.
%Konstanten sind namentlich gekennzeichnete Daten deren Werte nach der Deklaration und ihrer gleichzeitigen
%Definition/Initialisierung nicht mehr verändert werden.
%
\subsection{Integerliterale}
\label{p:2.22.1}
%
Dezimalliterale (Basis 10): \hfill
\begin{minipage}[t] {0.5\textwidth}
\begin{verbatim}
   100        // int;     100
   512L       // long;    512
   128053     // long; 128053
\end{verbatim}
\end{minipage}

Oktalliterale (Basis 8): \hfill
\begin{minipage}[t] {0.5\textwidth}
\begin{verbatim}
   020        // int;      16
   01000L     // long;    512
   0177       // int;     127
\end{verbatim}
\end{minipage}

Hexadezimalliterale (Basis 16): \hfill
\begin{minipage}[t] {0.5\textwidth}
\begin{verbatim}
   0x15       // int;      21
   0x200      // int;     512
   0x1ffffl   // long; 131071
\end{verbatim}
\end{minipage}\index{Literale!Integer}
%
%
\subsection{Gleitkommaliterale}
\label{p:2.22.2}
%
Gleitkommaliterale werden als \texttt{double} interpretiert solange
dies nicht anderweitig gekennzeichnet ist.
\index{Literale!Gleitkommazahl}\index{Gleitkommazahl}
\\
Einige Beispiele im folgenden:
\begin{minipage}[t] {0.6\textwidth}
\begin{verbatim}
   17631.0e-78
   1E+10              //   double: 10000000000
   1.                 //   double: 1
   .78                //   double: 0.78
   0.78
   -.2e-3             //   double: -0.0002
   -3.25f             //   single: -3.25
\end{verbatim}
\end{minipage}
%
%
\subsection{Zeichenliterale (Characterliterale)}
\label{p:2.22.3}
%
Die Characterliterale beinhaltet das Zeichen zwischen den zwei Apostrophen
\texttt{'} :
\begin{minipage}[t] {0.9\textwidth}
\begin{verbatim}
   'a', 'A', '@', '1' //  ASCII-Zeichen
   ' '                //    Leerzeichen
   '_'                //    Unterstreichung/Underscore
   '\''               //    Prime-Zeichen '
   '\\'               //    Backslash-Zeichen  \
   '\n'               //  neue Zeile
   '\0'               //  Nullzeichen NUL
\end{verbatim}
\end{minipage}\index{Literale!Character}\index{Ausgabe!neue Zeile}
%
%
\subsection{Zeichenkettenkonstanten (Stringkonstanten)}
\label{p:2.22.4}
%
Die Zeichenkette beinhaltet die Zeichen zwischen den beiden
Anführungszeichen \texttt{"} :
\begin{minipage}[t] {0.8\textwidth}
\begin{verbatim}
   "Hello World\n"    //    \n is C-Stil fuer neue Zeile (C++: endl)
   ""                 //    leere Zeichenkette
   "A"                //    String "A"
\end{verbatim}
\end{minipage}\index{Konstante!String}

Jede Zeichenkette wird automatisch mit dem
(Character-) Zeichen \verb|'\0'| abgeschlossen
(\textit{``Hey, hier h"ort der String auf!''}).
Daher ist \verb|'A'| ungleich \verb|"A"|, welches
sich aus den Zeichen  \verb|'A'| und \verb|'\0'|
zusammensetzt und somit 2~Byte zur Speicherung ben"otigt.
\includecode[firstline=4]{Ex224.cpp}{Länge von String und Character}
%
\index{Literale|)}
%
%
\section{Einige h"ohere Datentypen in C++}
\label{p:2.3}
%
Die Standardbibliothek und
die \textbf{S}tandard \textbf{T}emplate \textbf{L}ibrary (STL) stellen eine gro"se Auswahl
an komfortablen h"oheren Datenkonstrukten (genauer: \emph{Klassen} und \emph{Container}) zur Verf"ugung, welche
das Programmieren vereinfachen.
Die intensive Anwendung der Container f"ur eigene Datenstrukturen (und \emph{Klassen})
erfordert Kenntnisse in Klassenprogrammierung, Vererbung und  Templates, welche wir erst später behandeln werden.
Um aber schon mit diesen höheren Datentypen arbeiten zu können,  erfolgt hier ein kurze Einführung
in \texttt{string}, \texttt{complex},
\texttt{vector} und \texttt{valarray}.
Vertiefend und weiterf"uhrend seien dazu u.a.
\cite{KirchPrinz:2002:OOP, KuhlinsSchader:2002:DCS, SatirBrown:1995:PCP}
empfohlen.
%

\subsection{Die Klasse string}
\label{p:2.3.1}
Diese Standardklasse \texttt{string}~\cite[\S~18]{KirchPrinz:2002:OOP}
erlaubt eine komfortable Zeichenkettenverarbeitung
\includecode[firstline=4]{demoString.cpp}{Erste Schritte mit \texttt{string}}
%
Die Konvertierung von Zahlen in Strings und umgekehrt wird \ghref{http://www.cplusplus.com/articles/D9j2Nwbp/}{hier} erläutert und ist noch komfortabler mit
  \texttt{to\_string} und \texttt{stof}.
%
%\newpage

\subsection{Die Klasse complex}
\label{p:2.3.2}
Die Standardklasse \texttt{complex<T>}~\cite[p.~677]{KirchPrinz:2002:OOP}
erlaubt die Verwendung komplexer Zahlen \bspfile{demoComplex.cpp} in der gewohnten Weise.
Bei der Variablendeklaration mu"s der Datentyp der Komponenten der komplexen Zahl in
Dreiecksklammern angegeben werden. Hier machen eigentlich nur \verb|<double>| oder
\verb|<float>| einen Sinn.
%
\includecode[firstline=2]{demoComplex.cpp}{Erste Schritte mit \texttt{complex}}
%

\subsection{Die Klasse vector}
\label{p:2.3.4}
Die Containerklasse \texttt{vector<T>}~\cite[\S~5.1]{KuhlinsSchader:2002:DCS}
erlaubt die Verwendung von Vektoren \bspfile{demoVector.cpp} in vielf"altiger Weise, insbesondere
bei dynamisch ver"anderlichen Vektoren.
Bei der Variablendeklaration mu"s der Datentyp der Komponenten in
Dreiecksklammern angegeben werden. Wir werden meist \verb|<double>|,
\verb|<float>| oder \verb|<int>| verwenden, obwohl alle (korrekt implementierten) Klassen
hierf"ur benutzt werden d"urfen.
Die Klasse  \texttt{vector<T>} enh"alt keine Vektorarithmetik.
%
\subsubsection{Statischer Vektor}
Nachfolgender Code demonstriert den Einsatz eines statischen Vektors, d.h.,
die Anzahl der Vektorelemente is a~priori bekannt.
Mit \texttt{resize(\emph{laenge})} kann diese Anzahl der Elemente auf einen neuen Wert
ge"andert werden (dies ist eigentlich schon dynamisch). Die neuen Elemente m"ussen danach noch
mit Werten initialisiert werden.
%
\includecode[firstline=4]{demoVector.cpp}{Erste Schritte mit \texttt{vector} als statischem Vektor}
%

\subsubsection{Dynamischer Vektor}
Obiger Code behandelt Vektoren konstanter L"ange. Wenn die Anzahl der Vektorelemente a~priori unbekannt ist,
dann kann ein vorhandener Vektor durch Benutzung der Methode \texttt{push\_back(\emph{value})} um ein
weiteres Element mit dem Wert~\emph{value} verl"angert werden.
%
\includecode[firstline=2]{demoVector_dynamisch.cpp}{Erste Schritte mit \texttt{vector} als dynamischem Vektor}
%

\subsubsection{Ausgabe eines Vektor}
Leider k"onnen die Vektoren aus den vorherigen Beispielen nicht einfach mittels
\verb|cout << bb| ausgegeben werden. Ein solcher Versuch f"uhrt zu sehr kryptischen Fehlermeldungen
des Compilers. Es gibt drei prinzipielle M"oglichkeiten zur Ausgabe der Elemente eines
Vektor \verb|vector<int> bb|~:
\begin{enumerate}
  \item durch Ausgabe in einem \textrm{for}-Loop (konventionell),
  \item durch Kopieren auf den Ausgabestream (benutzt Algorithmus \texttt{copy} aus der STL),
  \item durch Definition einer Funktion f"ur den Ausgabeoperator \verb|<<| (Operator"uberladung).
\end{enumerate}
Alle 3 M"oglichkeiten werden in nachfolgendem Code demonstriert wobei jeweils exakt dieselbe Ausgabe
erzielt wird. Der Sprung auf die n"achste Ausgabezeile (\verb|cout << endl| nach jeder Variante)
wurde weggelassen.%\exfile{Ex234\_b.cpp}
Die Methoden \verb|begin()| und \verb|end()| zeigen auf das erste Elemente und das
hinterletzte Element (Diese beiden Methoden liefern Iteratoren zur"uck).
\includecode[firstline=2]{Ex234_b.cpp}{Ausgabe eines \texttt{vector}}
%

\subsection{\mbox{}$^{*}$Die Klasse valarray}
\label{p:2.3.3}
Die (nicht Standard) Klasse \texttt{valarray<T>}~\cite[p.~687]{KirchPrinz:2002:OOP}
erlaubt die Verwendung numerischer Vektoren \bspfile{demoValarray.cpp} wie sie von
\textsc{MatLab} bekannt sind. Insbesondere sind Vektorarithmetik und mathematische Funktionen vorhanden.
Bei der Variablendeklaration mu"s der Datentyp der Komponenten in
Dreiecksklammern angegeben werden. Wir werden \verb|<double>|,
\verb|<float>| oder \verb|<int>| verwenden.
%
\includecode[firstline=2]{demoValarray.cpp}{Vektorarithmetik mit \texttt{valarray}}
%
%Die Template-Klasse \texttt{valarray<T>} ist nicht in allen STL-Distributionen vorhanden.


























