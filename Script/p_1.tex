\chapter{Ein Schnelleinstieg}
\label{p:1}
%
\section{Was ist ein Programm und was nützt es \emph{mir}?}
\label{p:1.1}
%
Eigentlich kennt jeder bereits Programme, jedoch versteht
man oft verschiedene Inhalte darunter.
\begin{itemize}
 \item Parteiprogramm	\hfill $\Longleftrightarrow$ \hfill Ideen
 \item Theaterprogramm	\hfill $\Longleftrightarrow$ \hfill Ablaufplanung
 \item Musikpartitur	\hfill $\Longleftrightarrow$ \hfill strikte Anweisungsfolge
 \item Windowsprogramm	\hfill $\Longleftrightarrow$ \hfill
 			interaktive Aktion mit dem Computer
\end{itemize}
\textbf{Programmieren} ist das Lösen von Aufgaben auf dem Computer
mittels eigener Software und beinhaltet alle vier Teilaspekte in obiger
Liste.

Eine typische Übungsaufgabe beinhaltet folgenden Satz:

\fbox {\begin{minipage} {0.95\textwidth}
 \mbox{}\hfill \vdots \hfill\mbox{} \\
 Ändern [editieren] Sie das Quellfile [source file] entsprechend
 der Aufgabenstellung, übersetzen [compilieren] und testen
 Sie das Programm.
\end{minipage}}
\\[1.5ex]

% \pagebreak[4]
\centerline {\Large Was (?) soll ich machen ?}
\mbox{}\\[1ex]
%
\begin{tabular}{p{0.25\textwidth}@{\qquad}p{0.5\textwidth}@{\qquad}p{0.15\textwidth}}
 Idee, z.B.\   math.\  Verfahren & Im Kopf oder auf dem Papier. \newline(Was soll der Computer machen?) & Programmidee \\
 $\hspace{0.09\textwidth}\Downarrow$  \\
%
 Idee f"ur den Computer aufbereiten &
 	Entwurf.\newline (Wie kann der Computer die Idee realisieren?) & Struktogramm \\
 $\hspace{0.09\textwidth}\Downarrow$  \\
%
 Idee in einer Pro\-grammier\-sprache formulieren.
  & Quelltext/Quellfile  editieren.\newline
   (Welche Ausdr"ucke darf ich verwenden?) & Programmcode \\
  $\hspace{0.09\textwidth}\Downarrow$  \\
%
 Quellfile f"ur den Computer "ubersetzen
  & File compilieren [und linken]. \newline("Ubersetzung in Prozessorsprache)
  & ausf"uhrbares Programm
  \\
 $\hspace{0.09\textwidth}\Downarrow$  \\
%
 Programmcode ausf"uhren
 & Programm mit verschiedenen Datens"atzen testen
 & Programmtest
\end{tabular}
\\[2ex]

\begin{samepage}
Bemerkungen:
\begin{enumerate}
 \item Software = ausf"uhrbares Programm + Programmcode  + \textbf{Ideen}
 \item Der Lernproze{\ss}  beim Programmieren erfolgt typischerweise von
 unten nach oben in der vorangegangenen "Ubersicht.
 \item Als Mathematiker sind vorrangig Ihre mathematischen Kenntnisse und Ideen gefragt,
   jedoch ist deren \textbf{eigenständig}e Umsetzung ein großer Vorteil bei vielen
   Arbeitsplätzen.
 \item Mit \textbf{fundierten Kenntnisse}n in einer Programmiersprache fällt es recht leicht 
 weitere Programmiersprachen zu erlernen.
\end{enumerate}
\end{samepage}

\fbox {\begin{minipage} {0.95\textwidth}
 \underline {\textbf {Warnung} :} Das Programm auf dem Computer wird
 \textbf{genau das} ausf"uhren, was im Programmcode beschrieben ist!
\end{minipage}}

Typischer Anf"angerkommentar:
%%\begin{minipage}[t] {0.55\textwidth}
 \textit{Aber das habe ich doch ganz anders gemeint.}
%%\end{minipage}

\fbox {\begin{minipage} {0.95\textwidth}
 \underline {\textbf {Merke} :} Computer sind strohdumm!
 Erst die (korrekte und zuverl"assige) Software nutzt die
 M"oglichkeiten der Hardware.
\end{minipage}}

\paragraph{Warum denn C++, es gibt doch die viel bessere Programmiersprache XYZ!}
Der Streit über die beste, oder die bessere Programmiersprache ist
so alt wie es Programmiersprachen gibt. Folgende Gründe sprechen gegenwärtig für C++:
\begin{itemize}
	\item C++ erlaubt sowohl \textbf{strukturiert}e, als auch \textbf{objektorientiert}e Programmierung.
	\item Strukturierte Programmierung ist die Basis jeder Programmierung im
	 wissenschaftlich-technischen Bereich.
	\item Sie können in C++ reine C-Programme schreiben wie auch rein objektorientiert programmieren,
	d.h., es ist eine sehr gute Trainingsplattform.
	\item C++ erlaubt ein höheres \textbf{Abstraktionsniveau} beim Programmieren, d.h.,
	 ich muß mich nicht um jedes (fehleranfällige) informatische Detail kümmern.
	 Andererseits kann ich genau dies tun, falls nötig.
	\item C++ ist eine Compilersprache, keine Interpretersprache, und damit können
	die resultierenden Programme schnell sein.
	\item Die \ghref{http://gcc.gnu.org/}{Gnu}-Compiler für C++ sind für
	  alle gängigen (und mehr) Betriebssysteme,
	 insbesondere Linux, Windows, Mac-OS, \textbf{kostenlos} verfügbar.
	 Desgleichen gibt es gute, kostenlose Programmierentwicklungsumgebungen (IDE)
	 wie \ghref{http://www.codeblocks.org/}{CodeBlocks} auf diesen. \\
	 Die \ghref{https://clang.llvm.org/}{clang}-Compiler sind ebenfalls sehr zu empfehlen.
	\item Seit ca.\  20 Jahren ist C++ meist unter den Top-5 im Programmiersprachenranking
	vertreten, siehe  verschiedene Rankings wie
	\ghref{http://www.tiobe.com/index.php/content/paperinfo/tpci/index.html}{TIOBE Index} oder
	\ghref{http://pypl.github.io/PYPL.html}{PYPL}.
    Die Programmiertechniken der anderen Spitzensprachen lassen sich mit C++ ebenfalls realisieren.
    \item C++ mit seinen Bibliotheken ist sehr gut dokumentiert, siehe
    \ghref{http://www.cplusplus.com}{cplusplus.com}, \ghref{http://en.cppreference.com/w/}{cppreference.com} und 
    natürlich \ghref{https://stackoverflow.com/}{stackoverflow} für schwierigen Fälle.
    \item C++ wird weiterentwickelt, der neue
    \ghref{http://en.wikipedia.org/wiki/C++11}{C++11, C++14, C++17} Standard ist in den
    Compilern umgesetzt und C++20 ist in Arbeit. 
    Wir werden die Möglichkeiten des Standards C++11 und C++-17 an passender Stelle benutzen.
\end{itemize}
%
%
%
%\pagebreak[4]
\newpage
\section{Das ``Hello World'' - Programm in C++}
\label{p:1.2}
%
Wir beginnen mit dem einfachen ``Hello World''-Programm, welches
nur den String ``Hello World'' in einem Terminalfenster ausgeben wird. Damit läßt sich
schon überprüfen, ob der Compiler und die IDE korrekt arbeiten
(und Sie diese bedienen können).
%
%\includecode[linerange={12-17,30-31}]{HelloWorld.cpp}{Quelltext von Hello World}
\includecode[firstline=12]{HelloWorld.cpp}{Quelltext von Hello World}
%
Der simple Code im Listing~\ref{lst:HelloWorld.cpp} enthält schon einige
grundlegende Notwendigkeiten eines C++-Programmes:
%
\begin{itemize}
  \item Kommentare bis zum Zeilenende werden mit \verb| // | eingeleitet.\index{Kommentar!C++}
  \\
  Der C-Kommentar \verb|/*    */| kann auch in C++ verwendet werden.\index{Kommentar!C}
  \item Jedes Programm benötigt eine Funktion \verb|main()|, genannt Hauptprogramm\index{main()}.
  \begin{itemize}
	  \item \verb|int main()|  deklariert (ankündigen) das Hauptprogramm in Zeile~4.
	  \item Die geschweiften Klammern \verb|{ }| in Zeilen~5 und~8 begrenzen den
	  Funktionskörper der Funktion \verb|main|.
	  \item In Zeile 7 wird der Ausdruck\index{Ausdruck} \verb| return 0 |
	  durch den Strichpunkt \verb| ; | zu einer Anweisung\index{Anweisung}
	  im Programm. Diese spezielle Anweisung beendet das Hauptprogramm
	  mit dem Rückgabewert 0.
  \end{itemize}
  \item Die Ausgabe in Zeile~6 benutzt die I/O-Bibliotheken von C++.
  \begin{itemize}
    \item \verb|cout | ist ein Bezeichner für die Ausgabe im Terminal.
    \item \verb|<< |  leitet den nachfolgenden String auf den Ausgabestrom (\verb|cout|) um.
      Dies kann wiederholt in einer Anweisung geschehen.
    \item \verb|endl |  ist der Bezeichner für eine neue Zeile im Ausgabestrom.
    \item Die Preprocessor-Anweisung (beginnt mit \verb|#|) in Zeile~1 inkludiert
      das, vom Compiler mitgelieferte, Headerfile\index{Headerfile} \emph{iostream}
      in den Quelltext. Erst dadurch können Bezeichner wie \verb|cout | und
       \verb|endl | der I/O-Bibliothek benutzt werden.
    \item Ohne Zeile~2 müssen der Ausgabestrom etc.\   über die explizite Angabe
    des Namensraumes \verb|std| angegeben werden, also als \verb| std::cout |.
    Mit Zeile~2 wird automatisch der Namensraum \verb|std| berücksichtigt wodurch
    auch \verb|cout|  identifiziert wird.
  \end{itemize}
\end{itemize}
%

\underline{Quelltext eingeben und compilieren, Programm ausf"uhren}:
\begin{enumerate}
 \item Quellfile editieren. \index{Quellfile!editieren} \\
       \texttt{LINUX> geany HelloWorld.cpp}
 \item Quellfile compilieren. \index{Quellfile!compilieren}\index{Compilieren!g++} \\
	\texttt{LINUX> g++ HelloWorld.cpp}
 \item Programm ausf"uhren. \index{Programm!ausf\"uhren} \\
	\texttt{LINUX> a.out} \qquad oder \\
	\texttt{LINUX> ./a.out} \qquad oder \\
	\texttt{WIN98> ./a.exe}
\end{enumerate}
%
%
\underline{Tip zum Programmieren}: \\
 Es gibt (fast) immer mehr als eine M"oglichkeit, eine Idee im
 Computerprogramm zu realisieren. \\
 $\Longrightarrow$ Finden Sie Ihren eigenen Programmierstil
 	und verbessern Sie ihn laufend.
%
%
%
\newpage
\section{Interne Details beim Programmieren}
\label{p:1.3}
%
Der leicht ge"anderte Aufruf zum Compilieren	\\
\texttt{LINUX> g++ -v HelloWorld.cpp}		\\
erzeugt eine längere Bildschirmausgabe, welche mehrere
Phasen des Compilierens anzeigt.
Im Folgenden einige Tips, wie man sich diese einzelnen Phasen
anschauen kann, um den Ablauf besser zu verstehen:
%
\begin{enumerate}
  \renewcommand {\labelenumi}{\alph{enumi})}
  \item \emph{Preprocessing}: \index{Preprocessing}
  	Headerfiles\index{Headerfile}
	(\textit{*.h}, \textit{*} und \textit{*.hpp}) werden zum
	Quellfile hinzugef"ugt (+ Makrodefinitionen, bedingte Compilierung) \\
	\texttt{LINUX> g++ -E HelloWorld.cpp > HelloWorld.ii} \\
	Der Zusatz \quad\texttt{> HelloWorld.ii}\quad lenkt die Bildschirmausgabe
	in das File \textit{HelloWorld.ii}. Diese Datei \textit{HelloWorld.ii}
	kann mit einem Editor angesehen werden und ist ein langes C++
	Quelltextfile.
  \item "Ubersetzen in \emph{Assemblercode}: \index{Assembler}
  	Hier wird ein Quelltextfile in der
	(prozessorspezifischen) Programmiersprache Assembler erzeugt. \\
	\texttt{LINUX> g++ -S HelloWorld.cpp} \\
	Das entstandene File \textit{HelloWorld.s} kann mit dem
	Editor angesehen werden.
 \item \emph{Objektcode} erzeugen: \index{Objektcode}
 	Nunmehr wird ein File erzeugt, welches die direkten Steuerbefehle,
	d.h. Zahlen f"ur den Prozessor beinhaltet. \\
	\texttt{LINUX> g++ -c HelloWorld.cpp} \\
	Das File \textit{HelloWorld.o} kann nicht mehr im normalen
	Texteditor angesehen werden sondern mit \\
	\texttt{LINUX> xxd HelloWorld.o}
 \item \emph{Linken}: \index{Linken}\index{Objektcode}\index{Bibliothek}
 	Verbinden aller Objektfiles und notwendigen Bibliotheken
	zum ausf"uhrbaren Programm \textit{a.out}\enspace. \\
	\texttt{LINUX> g++ HelloWorld.o}
\end{enumerate}
%
%
%
\section{Bezeichnungen in der Vorlesung}
\label{p:1.4}
%
\begin{itemize}
 \item Kommandos in einer Befehlszeile unter LINUX:\index{Compilieren!g++}\\
 	\texttt{LINUX> g++ [-o myprog] file\_name.cpp} \\
	Die eckigen Klammern \verb| [  ] | markieren optionale
	Teile in Kommandos, Befehlen oder Definitionen.
	Jeder Filename besteht aus dem frei w"ahlbaren Basisnamen
	(\textit{file\_name}) und dem Suffix (\textit{.cpp}) welcher
	den Filetyp kennzeichnet.
 \item Einige Filetypen nach dem Suffix: \index{Suffix}

  \nopagebreak
  \begin{tabular}{l@{\qquad}p{0.6\textwidth}}
  	Suffix 		& Filetyp \\ \hline
	%\textit{.c}	& C-Quelltextfile \index{Quellfile}\\
	%\textit{.h}	& C-Headerfile (auch C++), Quellfile mit vordefinierten
				%Programmbausteinen \index{Headerfile}\\
	\textit{.cpp}
			& C++ -Quelltextfile\\
	\textit{.h} %,\textit{.hpp} [\textit{.hh}]
			& C++ -Headerfile\\
	\textit{.o}	& Objektfile \index{Objektfile}\\
	\textit{.a}	& Bibliotheksfile (Library) \index{Bibliothek} \\
        \textit{.exe}	& ausf"uhrbares Programm (unter Windows)
  \end{tabular}
  \item Ein Angabe wie \qquad
  	$\cdots$ \verb| < typ > | $\cdots$
	\qquad bedeutet, da{\ss}  dieser Platzhalter durch
	einen Ausdruck des entsprechenden Typs ersetzt werden mu\ss.
\end{itemize}
%
%
%
\newpage
\section{Integrierte Entwicklungsumgebungen}
\label{p:1.5}
%
Obwohl nicht unbedingt daf"ur n"otig, werden in der Programmierung h"aufig IDEs
(Integrated Development Environments) benutzt, welche Editor, Compiler, Linker und
Debugger - oft auch weitere Tools enthalten.
In der LV benutzen wir freie \ghref{http://gcc.gnu.org/}{Compiler} und
\ghref{http://www.gnu.org/}{-entwicklungstools},
insbesondere die Compiler basieren auf dem GNU-Projekt und funktionieren
unabhängig von Betriebssystem und Prozessortyp. Damit ist der von Ihnen geschriebene Code
portabel und läuft auch auf einem Supercomputer
(allerdings nutzt er diesen nicht wirklich aus, dazu sind weitere LVs n"otig).
Grundlage des Kurses sind die g++-Compiler ab Version 4.7.1, da diese auch den neuen C++11-Standard unterstützen.
Gegebenenfalls muß der Code mit der zusätzlichen Option \verb| -std=c++11 | übersetzt werden.

Wir werden unter Windows die IDE \ghref{http://www.codeblocks.org/}{\textbf{Code::Blocks}}
Version 16.01 [Stand: Feb.~2016]
benutzen welche auf den GNU-Compileren und -Werkzeugen basiert.
Dies erlaubt die Programmierung unter Windows ohne die Portabilit"at zu verlieren, da diese IDE
auch unter LINUX verf"ugbar ist.
Sie k"onnen diese Software auch einfach privat installieren,
siehe \ghref{http://www.codeblocks.org/downloads/26\#windows}{Download}
(nehmen Sie \texttt{codeblocks-16.01mingw-setup.exe}) und das
\ghref{http://www.codeblocks.org/user-manual}{Manual}.
Installieren Sie in jedem Fall \textbf{vorher} das Softwaredokumentationstool
\ghref{http://www.doxygen.org}{doxygen} (und \ghref{http://cppcheck.sourceforge.net/}{cppcheck}),
da es nur dann automatisch in die IDE eingebunden wird.

Da die Entwicklungsumgebung alles vereinfachen soll, ist vor dem ersten Hello-World-Programm
etwas mehr Arbeit zu investieren.
\begin{enumerate}
 \item \texttt{Code::Blocks} aufrufen: \\
   auf dem \textbf{Desktop} das Icon \textbf{Code::Blocks} anklicken.
 \item In \texttt{Code::Blocks} ein neues \emph{Projekt} anlegen: \\
   \textbf{File $\longrightarrow$ New $\longrightarrow$ Project}
   \begin{enumerate}
     \item Im sich "offnenden Fenster das Icon  \textbf{Console Application} anklicken und
        dann auf \textbf{Go} klicken.
     \item Bei der Auswahl der Programmiersprache \textbf{C++} anklicken und dann \textbf{Next}.
     \item Den \textbf{Projektitel} angeben - hier bitte das Namensschema \emph{bsp\_nr} mit
          \emph{nr} als Nummer der abzugebenden "Ubungsaufgabe einhalten. \\
          Den \textbf{Folder} (das Verzeichnis) ausw"ahlen in welchem das Projekt gespeichert
          werden soll. Darin wird dann automatisch ein Unterverzeichnis mit dem Projektnamen angelegt.
          \\  \textbf{Next} klicken.
     \item Die \emph{Debug} und die \emph{Release configuration} aktivieren.
     %Bei den \textbf{Compiler} und \textbf{configuration} angeben und darauf achten, da"s
           %das Projekt als \textbf{C++-Projekt} markiert ist.
           Auf \textbf{Finish} klicken.
     \item Im Workspace erscheint das neue Projekt \emph{bsp\_1} welches in seinen
          \emph{Sources} das File \emph{main.cpp} enth"alt.
           Auf dieses File klicken.
     \item Im Editor sehen Sie nun folgenden Programmtext: \\
\begin{minipage}{0.5\textwidth}
%
{\scriptsize
\begin{boxedverbatim}
#include <iostream>

using namespace std;

int main()
{
    cout << "Hello world!" << endl;
    return 0;
}
\end{boxedverbatim}
}
%
\end{minipage}
  \item Compilieren und Linken: \textbf{Build $\longrightarrow$ Build}
  \item Programm ausführen: \textbf{Build $\longrightarrow$ Run}
  \item Speichern dieser Datei: \textbf{File $\longrightarrow$ Save}
   \end{enumerate}
\end{enumerate}

\newpage
\section{Erste Schritte mit Variablen}
\label{p:1.6}
%
Wir beginnen mit dem einfachen ``Hello World''-Programm, welches
nur den String ``Hello World'' in einem Terminalfenster ausgibt.
Damit läßt sich schon überprüfen, ob der Compiler und die IDE korrekt arbeiten
(und Sie dies bedienen können).
Anschließend arbeiten wir mit ein paar einfachen Variablen.
%
%\includecode[linerange={12-17,30-31}]{HelloWorld_2.cpp}{Erweitertes ``Hello World''}
\includecode[firstline=2]{HelloWorld_2.cpp}{Erweitertes ``Hello World''}
%
Obiger Code enthält bekannte Teile aus dem Listing~\ref{lst:HelloWorld.cpp}, wir werden
kurz die neuen Zeilen erläutern:
\begin{itemize}
	\item[{[2]}] Die Deklarationen für den Datentyp (die Klasse) \texttt{std::string}
	  werden inkludiert. In Zeile~3 wird der Namensraum für \verb|std| freigegeben.
	\item[{[8]}] Ein \ghref{http://de.wikipedia.org/wiki/Integer_(Datentyp)\#Maximaler_Wertebereich_von_Integer}
	{ganzzahlige Variable}~\texttt{i} wird deklariert und darf damit in
	  den nachfolgenden Zeilen des Gültigkeitsbereichs zwischen \verb| { |
	  und  \verb| } | verwendet werden, also bis zur Zeile~22. Die Variable~\texttt{i}
	  kann ganzzahlige Werte aus $[-2^{-31},2^{-31}-1]$ annehmen.
    \item[{[10]}] Die Variable~\texttt{i} wird über den Terminal (Tastatureingabe)
     eingelesen. \verb| cin | und \verb| >> | sind Eingabestrom und -operator analog
     zur Ausgabe mit \verb| cout | und \verb| << |.
    \item[{[12]}] Die Variable~\texttt{i} wird gemeinsam mit einem beschreibenden String
     (Zeichenkette) ausgegeben.
    \item[{[14]}] Deklaration der Gleikommazahlen einfacher Genauigkeit
     (engl.: \ghref{http://de.wikipedia.org/wiki/IEEE_754\#Zahlenformate_und_andere_Festlegungen_des_IEEE-754-Standards}{single precision})
    \texttt{a} und \texttt{b}.
    \item[{[16]}] Einlesen von Werten für \texttt{a} und \texttt{b}. Ausgabe der
     Variablenwerte in Zeile~17.
    \item[{[19]}] Deklaration der Gleikommavariablen~\texttt{c}  und gleichzeitige Definition
    (Zuweisen eines Wertes) aus den Variablen \texttt{a} und \texttt{b}.
      Hierbei ist \verb| = | der Zuweisungsoperator und \verb| + | der Additionsoperator
      für Gleitkommazahlen.\index{Operator!Zuweisungs-}\index{Operator!Additions-}
    \item[{[20]}] Deklaration und gleichzeitige Definition eines konstanten (\texttt{const}) Strings~\texttt{ss}. Das Schlüsselwort legt fest, daß der \texttt{ss} zugewiesene Wert nicht mehr
      verändert werden kann.
    \item[{[21]}] Gemeinsame Ausgabe des konstanten Strings und der Gleikommavariablen.
    \item Mit \verb|{ }|  eingeschlossene Bereiche eines Quelltextes
	  definieren die Grenzen eines Gültigkeitsbereiches (scope)\index{Block}\index{scope}
	  von darin definierten Variablen.
\end{itemize}

Damit haben wir eine Grundlage, um uns in die Programmierung mit C++
schrittweise einzuarbeiten.
