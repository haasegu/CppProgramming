%%%%%%%%%%%%%%%%%%%%%%%%%%%%%%%%%%%%%%%%%%%%%%%%%%%%%%%%%%%%%%%%%%%%%%%%%%%%%%%%%%%%%%%%%%%%%%%%%%%%%%%%
%
\section{Casting in der Klassenhierarchie$\mbox{}^\ast$}
\label{sec:A4}
{\Large\bf   Nimm den neuen Code !!}
Castings (Konvertierungen) waren bislang stillschweigend in unseren Programmen
enthalten, wie in
\begin{lstlisting}[caption={Implizites Casting bei einfachen Datentypen.},label=lst:casting_implizit_1,
basicstyle=\scriptsize,numbers=left, numberstyle=\tiny, stepnumber=2, numbersep=5pt]
float salesPerson::payment() const
{
 double  dd = 15.373;
 int     ii = 5, kk;

 dd = ii;
 kk = dd;
}
\end{lstlisting}

%
Das erste Casting (\texttt{int} $\longrightarrow$ \texttt{double}) ist
problemlos, da die Nachkommastellen der Gleitkommazahl mit Nullen gef"ullt werden.
Im Gegensatz dazu gehen  bei dem zweiten Casting
(\texttt{double} $\longrightarrow$ \texttt{int})
s"amtliche Nachkommastellen verloren und bei ensprechenden Optionen (\verb|-Wall|)
gibt der Compiler eine Warnung aus. Beide Castings sind versteckt im Code
enthalten, daher nennt man sie \emph{implizit}\index{Casting!implizit}.
Mit einem \emph{explizit}en Casting\index{Casting!explizit}
(Cast-operator)\index{cast} der zweiten Zuweisung \\
\verb|  kk = (int)dd;   // C-Casting | \\ oder \\
\verb|  kk = static_cast<int>(dd);   // C++-Casting | \\
l"a"st sich die Warnung (beim \verb|g++|) abschalten.
%
%
\subsection{Implizites Casting}
\label{sec:A4.1}
%
Bei Klassen gibt es eine implizite Typkonvertierung nur in
Richtung abgeleitete Klasse zu Basisklasse vor dem Hintergrund, da"s
eine abgeleitete Klasse ein \emph{spezieller Typ} der Basisklasse ist
(also gemeinhin mehr Member/Eigenschafter beinhaltet als diese).
Alle Member der Basisklasse sind automatisch in der abgeleiteten Klasse enthalten, aber nicht umgekehrt.
Das folgende Beispiel 
% \exfile{A3/a3\_2.cpp} 
Die Abbilldung der Klassenhierarchie auf Seite~\pageref{class_hierarchy} 
veranschaulicht diesen Mechanismus des \emph{Upcasting}\index{Upcasting} und \emph{Downcasting}\index{Downcasting} gemeinsam mit nachfolgendem Beispiel.

\begin{lstlisting}[caption={Casting bei Klassen.},label=lst:casting_klassen_1,
basicstyle=\scriptsize,numbers=left, numberstyle=\tiny, stepnumber=2, numbersep=5pt]
...
 Worker aWage;
 salesPerson  aSale("Buddy Holly");

  aWage = aSale;       //   erlaubtes Upcasting
  aWage.payment();     //    (aber Provision ist futsch)

  salesPerson aSale2;
//  aSale2 = aWage;    //   nicht erlaubtes Downcasting
...
\end{lstlisting}

%
In Zeile 6 sind Name, Stundenlohn und Stundenanzahl (Members) des Angestellten \texttt{aWage}
als entsprechende Eintr"age des Verk"aufers \texttt{aSale} vorhanden.
Die Verkaufsprovison (spezielles Member) wird dabei nicht "ubernommen da diese in der Klasse Worker nicht vorkommt.
Umgekehrt m"u"ste der Compiler eine unkommentierte Zeile~10 ablehnen, denn
woher soll er (implizit!!) eine vern"unftige Verkaufsprovision aus den
vorhandenen Daten des Angestellten ableiten
(die Klasse \texttt{Worker} besitzt weniger Member  als \texttt{salesPerson})?
N"aheres zu Konvertierungskonstruktoren und -funktionen zwischen Klassen steht in
\cite[\S9 und \S19]{KirchPrinz:2002:OOP}\index{Casting},
\cite[\S9.4.4]{Schmaranz:2002:SCP},
\cite[\S16.5]{SchaderKuhlins:1998:PCP}.
%
%
\subsection{Casting von Klassenpointern}
\label{sec:A4.2}
%
Die demonstrierte, implizite Typkonvertierung erlaubt uns, da"s ein mit einer
Basisklasse typisierter Pointer sowohl auf Instanzen der Basisklasse als auch auf
Instanzen davon abgeleiteter Klassen zeigt.

Dies, und mi"sbr"auchliche Benutzung
demonstrieren wir wiederum an einem
% Beispiel\exfile{A3/a3\_2.cpp}:\label{code:a3_2}
% v_9b_cast/main.cpp
\begin{lstlisting}[caption={Casting bei Klassenpointern.},label=lst:casting_klassen_2,
basicstyle=\scriptsize,numbers=left, numberstyle=\tiny, stepnumber=2, numbersep=5pt]
//				a3_2.cpp
...
 salesPerson  aSale("Buddy Holly");
...
 salesPerson *salePtr;		//	nun das ganze mit Pointern
 Worker *wagePtr;

 salePtr = & aSale;
 wagePtr = & aSale;		//	Basisklassenpointer

 salePtr->setSales(1000);	// salesPerson :: setSales
//  wagePtr->setSales(1000);	// WagePerson  :: setSales  gibt es nicht !!

 salePtr->payment();		// salesPerson ::payment
 wagePtr->payment();		// Worker::payment

//	und nun schmutzig mit Pointern
  cout << endl << "         schmutzig mit Pointern" << endl;

  //salePtr = wagePrt;			// geht nicht
  salePtr = (salesPerson *) wagePtr;	// expizite Typkonvertierung, gefaehrlich !!
  salePtr->payment();

//	und nun  g a n z  schmutzig mit Pointern
  cout << endl << "         g a n z   schmutzig mit Pointern" << endl;

  Employee *empPtr = &simple;	// --> einfacher Employee
  salesPerson *sale_Ptr;

  sale_Ptr = (salesPerson *)empPtr;	// erlaubt, aber inkorrekt
  sale_Ptr->setComission(0.06);		// nunmehr erlaubt, aber katastrophal
  sale_Ptr->payment();
...
\end{lstlisting}
%
%  dynamic_cast mit Pointer:  http://en.cppreference.com/w/cpp/language/dynamic_cast
%
% dynamic_cast  mit Referenz:  http://www.cplusplus.com/reference/typeinfo/bad_cast/
%  
% Upcasting :http://www.bogotobogo.com/cplusplus/upcasting_downcasting.php

%
Wie in Zeile 12 zu sehen ist, k"onnen nur die Methoden der Klasse aufgerufen werden,
f"ur welche der Pointer typisiert ist - obwohl er auf eine Instanz zeigt, f"ur
welche die Methode \verb|setSales| deklariert ist.
Heikel (und verboten geh"ort) ist die  Pointerkonvertierung
in Zeile~21 da die Berechnungsmethode in Zeile~22 auf nicht vorhandene Member
zugreifen k"onnte. Hier funktioniert es nur deshalb, da \verb|wagePtr| auf eine
Instanz der Klasse \verb|salesPerson| verweist (Zeile~9).

Die Katastrophe tritt dann aber in den Zeilen 30-32 ein, da hier
wirklich auf nicht vorhandene Member zugegriffen wird.
Im g"unstigsten Fall ist die Berechnung in Zeile~32 falsch, wenn man Pech hat,
dann st"urzt das Programm unkontrolliert ab.

Solche Pointerkonvertierungen von Basisklasse zu abgeleiteter Klasse
sind unn"otig, gef"ahrlich und zeugen von einem schlechten Klassendesign.

Wenn man schon Pointer (oder Referenzen) auf Klassen casten muß, dann sollte man 
\verb| dynamic_cast<class*>(class_ptr); |\index{Cast!dynamic\_cast}  benutzen, 
welches zur Laufzeit die Zulässigkeit des Castings prüft und bei Mißerfolg einen 
Nullpointer zurückgibt (oder eine \texttt{bad\_cast}-Exception wirft).
%
%
\subsection{Nutzung von Basisklassenpointern}
\label{sec:A4.3}
%
Nat"urlich stellt sich die Frage, wozu man Basisklasenpointer gebrauchen kann.
Eine Teilantwort ist, da"s es damit z.B., m"oglich ist alle verschiedenen
Angestellten (\texttt{Employee}) in einem Array zu speichern und
dann Aktionen nur noch auf die einzelnen Feldelemente anzuwenden.
Dieser Ansatz wird in im nachfolgenden Beispiel
demonstriert.
% \exfile{A3/a3\_3.cpp}:
\begin{lstlisting}[caption={Nutzung von Klassenpointern.},label=lst:casting_klassen_3,
basicstyle=\scriptsize,numbers=left, numberstyle=\tiny, stepnumber=2, numbersep=5pt]
//		a3_3.cpp
#include <iostream>
#include "employ.hpp"
using namespace std;

int main()
{
 Employee  simple("Josef Gruber");

 Worker hilf("Heiko");
 hilf.setWage(20); hilf.setHours(7.8);

 salesPerson emp("Gundolf Haase");
 emp.setWage(hilf.getWage());  emp.setHours(hilf.getHours());
 emp.setComission(0.05); emp.setSales(10000.0);

 Manager man("Max Planck");
 man.setSalary(1000.0);

 const int N=4;
 Employee* liste[N];		// array von Pointern auf Employee
 int i;

 liste[0] = &simple;
 liste[1] = &hilf;
 liste[2] = &emp;
 liste[3] = &man;

 cout << endl << "   Nur die Namen ausgeben" << endl;
 for (i=0; i<N; i++)
  {
    liste[i]->info();
  }

 cout << endl << endl << "   Name und (spezifisches) Gehalt klappen nicht" << endl;
 for (i=0; i<N; i++)
  {
    liste[i]->payment();	//	==> Ausweg: Virtuelle Methoden
  }

 return 0;
}
\end{lstlisting}
% \label{code:a3_3}
Die Namensausgabe in Zeile 32 funktioniert so wie geplant, jedoch
wird in Zeile 38 stets die Methode
\verb|Employee :: payment()| aufgerufen, welche jedoch keine
geeignete Gehaltsberechnungsroutine zur Verf"ugung hat.
Die Definition einer Methode \texttt{payment()} f"ur die Basisklasse
\texttt{Employee} l"ost das Problem auch nicht,
da in Zeile 38 die verschiedenen Besch"aftigungsverh"altnisse
\emph{a priori} nicht bekannt sind.
Aber gerade diese Unterschiede sind das Spezielle in den abgeleiteten Klassen.

Nat"urlich bietet C++ einen Ausweg aus dem Dilemma - die \emph{virtuellen Methoden}
mit ihren \emph{dynamischen Bindungen}.
\index{Methoden!virtuell}\index{dynamische Bindung}
%
%
\subsection{Einige Bemerkungen zum Casting}
\label{sec:A4.4}
An Stelle des C-Casts \verb|(Typ) Ausdruck| wie in \verb|(double) idx|
sollte man die vier in C++ enthaltenen Casts benutzen, welche ein spezifischeres
Casting erlauben, leichter im Programmcode auffindbar sind und es dem
Compiler auch erlauben, bestimmte Casts abzulehnen.\index{Cast!C-Cast}
%
\begin{itemize}
 \item \verb|static_cast<Typ>(Ausdruck)|\index{Cast!static\_cast}
 ersetzt das bekannte C-Cast, also w"urde obiges Beispiel zu
 \verb|static_cast<double>(idx)|
%
 \item \verb|const_cast<Typ>(Ausdruck)|\index{Cast!const\_cast} erlaubt die
 Beseitigung der Konstantheit eines Objektes. Eine Anwendung daf"ur ist der
 Aufruf einer Funktion, welche nichtkonstante Objekte in der
 Parameterliste erwartet, diese aber nicht ver"andert.
 (Das Interface dieser Funktion ist schlecht designed)
\begin{verbatim}
void Print(Studenten&);
...
int main()
{
 const Studenten arni("Arni","Schwarz",89989, 787);
 Print(const_cast<Studenten>(arni));
}
\end{verbatim}
Vorsicht vor Mi"sbrauch wie in folgendem Code, in welchem eine Konstante pl"otzlich einen
anderen Wert hat.
% \begin{verbatim}
% void printInt(const int& v)
% {
%   cout << v << endl;
% }
%
% int main () {
%
%   const int a = 5;
%   int& j = const_cast<int&>(a);
%   j = -3;
%   cout << " a = " << a << "  i = "<< j << endl;
%   cout << " a = "; printInt(a);
%   cout << " i = "; printInt(j);
%
%   return 0;
% }
% \end{verbatim}
% %
 \item \verb|dynamic_cast<Typ>(Ausdruck)|\index{Cast!dynamic\_cast} dient
 der sicheren Umformung von Pointern und Referenzen
 in einer Vererbungshierarchie und zwar nach unten
 (abgeleitete Klasse) oder zwischen benachbarten Typen.
 So w"are das exakte Casting in Zeile~21 des Codes auf Seite~\pageref{code:a3_2}:
 \verb|salePtr = dynamic_cast<salesPerson*>(wagePtr);|~.
 Falls das Casting (zur Laufzeit!!) nicht erfolgreich ist wird ein Nullzeiger
 zur"uckgegeben bzw. eine Exception ausgeworfen bei Referenzen.
%
 \item \verb|reinterpret_cast<Typ>(Ausdruck)|\index{Cast!reinterpret\_cast}
 wird f"ur Umwandlungen benutzt deren Ergebnis fast immer implementationsabh"angig ist.
 Meist wird es zur Umwandlung von Funktionspointern benutzt.
\end{itemize}
%
Obige Erl"auterungen und weitere Beispiele zu diesen Casts sind in
\cite[\S1.2]{Meyers:1997:MEC}
zu finden. Mehr Beispiele und Bemerkungen zur Typ"uberpr"ufung der Casts,
siehe
\cite[p.246f]{Schmaranz:2002:SCP}.


