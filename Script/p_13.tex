\chapter{Tips und Tricks}
\label{p:13}
%
%

%
%
\section{Pr"aprozessorbefehle}
\label{p:13.1}
%
%
Wir kennen bereits die Pr"aprozessoranweisung
\index{Pr\"aprozessor}\index{Compilieren!bedingtes}\index{\#include} 
\\[0.5ex]
\verb|#include <cmath>|
\\[0.5ex]
welche vor dem eigentlichen Compilieren den Inhalt des Files 
\textit{cmath} an der entsprechenden Stelle im Quellfile einf"ugt. 
Analog k"onnen bestimmte Teile des Quelltextes beim Compilieren 
eingebunden oder ignoriert werden, je nach Abh"angigkeit des 
Tests %(analog einer Alternative wie in \S\ref{sec:4.3}) 
welcher mit einer Pr"aprozessorvariablen durchgef"uhrt wird. 
\bspfile{preproc.cpp}

Variablen des Pr"aprozessors werden mittels
\index{\#define} 
\\[0.5ex]
\verb|#define MY_DEBUG|
\\[0.5ex]
definiert und wir k"onnen auch testen, ob sie definiert sind:
\index{\#ifdef}
\\[0.5ex]
\verb|#ifdef MY_DEBUG|	\\
\verb|  cout << "Im Debug-Modus" << endl;|	\\
\verb|#endif|
\\[0.5ex]
Analog kann mit 
\index{\#ifndef}
\\[0.5ex]
\verb|#ifndef MY_DEBUG|	\\
\verb|#define MY_DEBUG|	\\
\verb|#endif|
\\[0.5ex] 
zun"achst getestet werden, ob die Variable \verb|MY_DEBUG| 
bereits definiert wurde. Falls nicht, dann wird sie eben jetzt 
definiert. 
\\ 
Obige Technik wird/wurde h"aufig im Header Guarding benutzt um zu verhindern, da"s 
die Deklarationen eines Headerfiles mehrfach in denselben 
Quelltext eingebunden werden. 
Dieses Header Guarding wurde in C++11 durch die Pragma-Direktive 
\verb|#pragma once| abgelöst.

Einer Pr"aprozessorvariablen kann auch ein Wert zugewiesen werden 
\\[0.5ex]
\verb|#define SEE_PI 5|
\\[0.5ex]
welcher anschlie"send in Pr"aprozessortests (oder im Programm 
als Konstante) benutzt werden kann:
\index{\#if} 
\\[0.5ex]
\verb|#if (SEE_PI==5)|	\\
\verb|  cout << " PI = " << M_PI << endl;|	\\
\verb|#else|	\\
\verb|// leer oder Anweisungen|	\\
\verb|#endif|
\\[0.5ex] 
Eine weitere Anwendung besteht in der Zuweisung eines Wertes zu einer 
Pr"aprozessorvariablen, falls diese noch nicht definiert wurde.
\\[0.5ex]
\verb|#ifndef M_PI|	\\
\verb|#define M_PI 3.14159|	\\
\verb|#endif|

Es können auch Präprozessor-Makros mit Parametern definiert werden
allerdings ist es weniger fehleranfällig vorhandene Funktionen zu nutzen oder 
eine kurze Templatefunktion in C++ zu schreiben.
Der Autor fand ein sinnvolles, nicht als C++-Funktion realisierbares, Makro\bspfile{preproc.cpp}
\\[0.5ex]
\verb|#define what_is(x) cerr << #x << " is " << (x) << endl;|
\\[0.5ex]
welches beim Aufruf \verb|what_is(N)| sowohl den Variablennamen als auch dessen Wert ausgibt.
%
%Desweiteren k"onnen Makros mit Parametern definiert 
%\\[0.5ex]
%\verb|#define MAX(x,y) (x>y ? x : y)|
%\\[0.5ex]
%und im Quelltext verwendet werden. 
%\\[0.5ex]
%\verb|  cout << MAX(1.456 , a) << endl;|
%\\[0.5ex]
\\
Mehr "uber Pr"aprozessorbefehle ist u.a.\  in 
\cite{Gode:1998:ANK} und \cite[\S A.11]{Stroustrup:2000:CPP} 
zu finden. 
%
%
\section{Zeitmessung im Programm}
\label{p:13.2}
%
Zum Umfang von C++ geh"oren einige Funktionen, welche es erlauben 
die Laufzeit bestimmter Programmabschnitte (oder des gesamten Codes) 
zu ermitteln. Die alte C-Funktion \texttt{clock()}\index{clock()} sollte nicht mehr verwendet werden
da diese auf Multi-Core-CPUs nicht die WallClockTime misst. 
%Die entsprechenden Deklarationen werden im Headerfile 
%\textit{ctime} bereitgestellt.\index{ctime!clock()}\index{ctime!CLOCKS\_PER\_SEC}\index{Zeitmessung}
%Die Zeitmessung mit \texttt{clock()} ist nur für sequentielle Programme hilfreich, 
Stattdessen sollten Klassen und Methoden \index{system\_clock} 
aus dem Header \verb|<chrono>|\index{chrono} verwendet werden.

%
%\includecode[linerange={15-19,25-25,42-43,51-51,66-67,74-75}]{Ex1121.cpp}
\includecode[linerange={10-10,16-21,38-39,49-49,66-66,71-72}]{Ex1121.cpp}
{Laufzeitmessung mit \emph{chrono}}
%
Es k"onnen beliebig viele Zeitmessungen im Programm erfolgen 
(irgendwann verlangsamen diese aber ihrerseits das Programm!). 
Jede dieser Zeitmessungen ben"otigt einen Start und ein Ende, 
allerdings k"onnen die Zeiten verschiedener Messungen 
akkumuliert werden (einfach addieren).

Im File \textit{Ex1121.cpp} \bspfile{Ex1121.cpp} 
wird der Funktionswert eines Polynoms vom Grade~20 
an der Stelle~$x$ d.h., 
$s = \sum_{k=0}^{20} a_k \cdot x^k$, berechnet. 
%Die 21~Koeffizienten $a_k$ und der Wert~$x$ werden im File 
%\textit{input.1121} bereitgestellt. 
Der Funktionswert wird auf zwei, mathematisch identische, Weisen im Programm berechnet. 
Variante~1 benutzt die Funktion \verb|pow|, w"ahrend 
Variante~2 den Wert von $x^k$ durch fortw"ahrende Multiplikation berechnet 
(\ghref{https://baptiste-wicht.com/posts/2017/09/cpp11-performance-tip-when-to-use-std-pow.html}{Implementierungsvergleich 2017} und 
\ghref{https://stackoverflow.com/questions/2940367/what-is-more-efficient-using-pow-to-square-or-just-multiply-it-with-itself}{2021}).  
\index{pow}

Das unterschiedliche Laufzeitverhalten (Ursache~!?)
kann nunmehr durch 
Zeitmessung belegt werden und durch fortschreitende Aktivierung 
von Compileroptionen (\ghref{https://kristerw.github.io/2021/10/19/fast-math/}{\texttt{-ffast-math}}) 
zur Pro\-gramm\-op\-ti\-mie\-rung etwas verbessert werden, z.B. 
\\[0.5ex]
\verb|LINUX>  g++ Ex1121.cpp|	\\
\verb|LINUX>  g++ -O Ex1121.cpp|	\\
\verb|LINUX>  g++ -O3 Ex1121.cpp|	\\
\verb|LINUX>  g++ -O3 -ffast-math Ex1121.cpp|	
\\[0.5ex]
Der Programmstart erfolgt jeweils mittels 
\\[0.5ex]
\verb|LINUX>  a.out|
%\verb|LINUX>  a.out < input.1121|
%
%

Kleine, selbst geschriebene Funktionen können diese Laufzeitstoppung 
analog \texttt{tic()} und \texttt{toc()} in Matlanb gestalten.
Hierzu das Headerfile \emph{timing.h} \bspfile{utils/timing.h} lokal herunterladen 
und wie folgt in den eigenen Code inkludieren.

\includecode[linerange={10-10,16-17,34-35,45-45,62-62,67-68}]{Ex1121_mod.cpp}{Laufzeitmessung mit \emph{timings.h}}
%
%
\section{Profiling}
\label{p:13.3}
%
Nat"urlich k"onnte man in einem Programm die Zeitmessung in jede 
Funktion schreiben um das Laufzeitverhalten der Funktionen und Methoden 
zu ermitteln. 
Dies ist aber nicht n"otig, da viele Entwicklungsumgebungen 
bereits Werkzeuge zur Leistungsanalyse (performance analysis), 
dem\index{Profiling} \textbf{Profiling} bereitstellen. 
Darin wird mindestens die in den Funktionen verbrachte Zeit und die 
Anzahl der Funktionsaufrufe (oft graphisch) ausgegeben. 
Manchmal l"a"st sich dies bis auf einzelne Quelltextzeilen aufl"osen. 
Neben den professionellen (und kostenpflichtigen) Profiler- und 
Debuggingwerkzeugen sind unter LINUX/UNIX  auch 
einfache (und kostenlose) Kommandos daf"ur verf"ugbar.
\index{Compilieren!g++} 
\\[0.5ex]
\verb|LINUX>  g++ -pg Jacobi.cpp matvec.cpp|	\\
\verb|LINUX>  a.out|	\\
\verb|LINUX>  gprof -b a.out > out|	\\
\verb|LINUX>  less out|	
\\[0.5ex]
Der Compilerschalter \verb|-pg| bringt einige Zusatzfunktionen im 
Programm unter, soda"s nach dem Programmlauf das Laufzeitverhalten 
durch \verb|gprof| analysiert werden kann. 
Der letzte Befehl (kann auch ein Editor sein) 
zeigt die umgeleitetete Ausgabe dieser Analyse auf dem Bildschirm an. 
%
%
\section{Debugging}
\label{sec:11.4}
%
Oft ist es notwendig den Programmablauf schrittweise zu verfolgen und sich 
gegebenenfalls Variablenwerte etc.\  zu Kontrollzwecken ausgeben zu lassen. 
Neben der stets funktionierenden, jedoch nervt"otenden, Methode
\index{Debugging} 

\verb| ...|	\\
\verb| cout << "AA " << variable << endl; |	\\
\verb| ...|	\\
\verb| cout << "BB " << variable << endl; |	\\
\verb| ...|

sind oft professionelle Debuggingwerkzeuge verf"ugbar. 
Hier sei wiederum ein (kostenfreies) Programm unter LINUX vorgestellt. 
\index{Compilieren!g++}
\\[0.5ex]
\verb|LINUX>  g++ -g Ex1121.cpp|	\\
\verb|LINUX>  ddd a.out &|	
\\[0.5ex]
Die Handhabung der verschiedenen Debugger unterscheidet sich sehr stark. 
Beim ddd-Debugger kann mit \verb|set args < input.1121| das 
Eingabefile angegeben werden und mit \verb|run| wird 
der Testlauf gestartet, welcher an vorher gesetzten Break-Punkten 
angehalten wird. Dort kann dann in aller Ruhe das Programm anhand des 
Quellcodes schrittweise verfolgt werden. 
 
%
%
\section{Einige Compileroptionen}
\label{sec:11.5}
%
Nachfolgende Compileroptionen beziehen sich auf den GNU-Compiler g++, jedoch 
sind die Optionen bei anderen Compilern teilweise identisch oder "ahnlich.
\begin{itemize}
 \item Debugging und Warnungen, welche viel Debuggingarbeit ersparen: \\
 	\texttt{g++ -g -Wall -Wextra -pedantic -Wswitch-default -Wmissing-declarations -Wfloat-equal -Wundef -Wredundant-decls -Wuninitialized -Winit-self -Wshadow -Wparentheses -Wunreachable-code}
 \item Zus"atzliche Optionen f"ur Klassenhierarchien nach \cite{Meyers:1998:ECP,Meyers:1997:MEC}:
 	\texttt{-Weffc++ -Woverloaded-virtual}
 \item Standardoptimierung:
 	\texttt{g++ -O}
 \item Bessere Optimierung:
 	\texttt{-Ofast}
 \item Den \texttt{assert()}-Test ausschalten:
     \texttt{-DNDEBUG}
\end{itemize}
 
 
%
%
\section{Numerik in C++}
\label{sec:11.6}
%
Die \ghref{https://en.cppreference.com/w/cpp/numeric}{numerische Bibliothek} von C++ wird 
kontinuierliche erweitert. 
Für uns von Interesse sind 
\begin{itemize}
 \item die üblichen \ghref{https://en.cppreference.com/w/cpp/numeric/math}{mathematischen Funktionen} inkl.\  der $\Gamma$-Funktion;
 \item \ghref{https://en.cppreference.com/w/cpp/numeric/special_functions}{spezielle} math. Funktionen, wie der Bessel-Funktion; 
 \item die gelisteten \emph{Numeric Operations} der STL, wie \texttt{accumulate} oder \texttt{inner\_product};
 \item Bitmanipulationen;
 \item die Möglichkeit, über die \ghref{https://en.cppreference.com/w/cpp/numeric/fenv}{Floating-point} environment auch einige 
 Execptions bei Berechnungen abzufragen 
 (Overflow, \ghref{https://en.cppreference.com/w/cpp/numeric/fenv/FE_exceptions}{DivisionByZero}, etc.);
 \item ein für Numerik optimierter Vektor \ghref{https://en.cppreference.com/w/cpp/numeric/valarray}{\texttt{valarray}}.
\end{itemize}
%

Daneben gibt es die \ghref{https://www.boost.org/}{\emph{boost-Library}}
mit einer überwältigenden \ghref{https://www.boost.org/doc/libs/}{Funktionalität}.
Die Boost ist auch ein Experimentierfeld für Datenstrukuren und Algorithmen welche bei Bedarf in den C++-Standard übernommen werden. 
\\
So sind in \ghref{https://www.boost.org/doc/libs/1_76_0/libs/graph/doc/index.html}{\emph{graph}} Datenstrukturen und viele 
Graphalgorithmen enthalten, neben weiteren mathemtischen Bibliotheken.

Für klassische lineare Algebra sei auf die Bibliotheken 
\ghref{http://www.netlib.org/blas/}{BLAS} (die optimierte OpenBLAS) und 
\ghref{https://www.netlib.org/lapack/}{LAPACK} mit den entsprechenden C/C++-Interfaces verwiesen.

%
%\pagebreak
%
\section{Zufallszahlen}
\label{sec:13.7}
Zur Zufallszahlengenerierung sind neben den alten C-Funktionen in \verb|<cstdlib>| in C++ 
viele Zufallszahlengeneratoren in  \ghref{https://en.cppreference.com/w/cpp/header/random}{\texttt{<random>}} verfügbar.\index{random}

Examplarisch\bspfile{random/main.cpp} demonstrieren wir die Generierung von LOOPS gleichverteilter ganzzahliger (Pseudo)-Zufallszahlen 
aus einem Intervall [ANF, ENDE] auf drei verschiedene Arten unter Nutzung von  
\ghref{https://en.cppreference.com/w/cpp/numeric/random/rand}{\texttt{rand()}},
\ghref{http://www.cplusplus.com/reference/random/linear_congruential_engine/operator()}{\texttt{minstd\_rand0}}, 
\ghref{https://en.cppreference.com/w/cpp/numeric/random/uniform_int_distribution}{\texttt{uniform\_int\_distribution}}.


%\paragraph{Zufallszahlen C-like}
\includecode[linerange={4-5,29-29,31-40}]{random/main.cpp}{Zufallszahlen C-like}
\includecode[linerange={5-5,7-7,42-42,47-56}]{random/main.cpp}{Zufallszahlen C++}
\includecode[linerange={7-7,58-58,62-73}]{random/main.cpp}{komfortable Zufallszahlen in C++}

 
