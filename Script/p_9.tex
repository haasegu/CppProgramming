\chapter{Erste Schritte mit Klassen}
\label{p:9}
%
Member -- Methoden; Kapselung; Kopierkonstruktor; Zuweisungsoperator

Eine Klasse ist ein Konzept, welches Daten gemeinsam mit Funktionen auf diesen Daten definiert.
Daher sind diese Funktionen an ein Objekt dieser Klasse gebunden und diese Art des
Programmierens wird \emph{objektorientiert} (OOP) genannt.
Ihre wahre Stärke spielen Klassen im Weiteren bei Klassenhierarchien und beim
Polymorphismus in \S\ref{p:12} aus.

Um klarer zwischen strukturierter und objektorientierten Konzepten zu unterscheiden werden
folgende Begriffe benutzt:
\begin{itemize}
    \item Daten in einer Klasse $\longrightarrow$ \emph{Member}\index{Member} (auch \emph{Eigenschaften}),
    \item Funktion in einer Klasse $\longrightarrow$ \emph{Methode}\index{Methode}
    \item Variable vom Typ der Klasse  $\longrightarrow$ \emph{Instanz}\index{Instanz}
    (auch \emph{Objekt}) der Klasse.
    \item \emph{Konstruktoren} dienen der Erstinitialisierung von Membern bei Deklaration einer Instanz.
    \item Der \emph{Destruktor} gibt Ressourcen wieder frei, wenn der Gültigkeitsbereich (scope)
      einer Instanz endet.
\end{itemize}
Natürlich wird strukturierte Programmierung innerhalb der Methoden wieder benötigt.

%%%%%%%%%%%%%%%%%%%%%%%%%%%%%%%%%%%%%%%%%%%%%%%%%%%%%%%%%%%%%%%%%%%%%%%%%%%%%%%%%%%%%%%%%%%%
\section{Unsere Klasse \texttt{Komplex}}
\label{p:9.1}
%
Wir führen weitere Begriffe an Hand der Beispielklasse \texttt{Komplex} ein, welche
eine komplexe Zahl samt der entsprechenden Funktionalität speichert.
Es gibt gibt bereits eine Template-Klasse
\ghref{http://www.cplusplus.com/reference/complex/complex/?kw=complex}{\texttt{complex}}
mit dieser Funktionalität, aber wir bezwecken hier eine einfache Einführung in Klassen.
Aus Gründen des besseren Verständnisses werden die grundlegend notwendigen Methoden
deklariert, obwohl diese vom Compiler auch automatisch angelegt würden.

\includecode[linerange={5-34},label=lst:komplex_main.cpp]{komplex/main.cpp}{Nutzung unserer Klasse Komplex}
%
Mit unserer Klasse \texttt{Komplex} soll das Programm in Listing~\ref{lst:komplex_main.cpp} funktionieren.
\begin{itemize}
    \item \textbf{Zeilen 6-8}: Deklaration (und Initialisierung) dreier Instanzen von \texttt{Komplex}.
    \item \textbf{Zeilen 9,15,17}: Addition zweier Instanzen und Zuweisung des Ergebnisses an eine dritte Instanz.
    \item \textbf{Zeilen 11,12,18,22}: Ausgabe jeweils einer Instanz über die Standardausgabe.
    \item \textbf{Zeile 24}: Ende des Gültigkeitsbereichs der drei Instanzen,
    impliziter dreifacher Aufruf des Klassendestruktors.
\end{itemize}
Jede komplexe Zahl besteht aus Real- und Imaginärteil, welche wir als Member speichern müssen.
Die Methoden der Klasse ergeben sich erstmal aus den Anforderungen des Listing.
Das Projekt besteht aus dem Hauptprogramm \emph{main.cpp}, dem Headerfile unserer Klasse
\emph{komplex.h} mit allen Deklarationen und dem Sourcefile \emph{komplex.cpp} mit den
Implementierungen der Methoden unserer Klasse.
Die Deklaration der Klasse kann GUI-unterstützt in codeblocks erfolgen
(File $\rightarrow$ New $\rightarrow$ Class), allerdings werden für die Methoden nur die
formalen Hüllen bereitgestellt - nicht die Implementierung!

\includecode[linerange={1-4,8-33,70-81,98-98},label=lst:komplex_komplex.h_a]{komplex/komplex.h}
{Teil des Headerfile für \texttt{Komplex}}
%
\begin{itemize}
    \item \textbf{Zeilen 1,2,39}: Die \emph{Header Guards} sind Präprozessoranweisungen
   welche verhindern, daß die Deklarationen (und gegebenfalls auch Definitionen) nur einmal
   in jedes Quellfile eingebunden werden. Unbedingt benutzen!
   \item \textbf{Zeilen 5,6,38} sind der Rahmen für die Deklaration der Klasse.
   \item Einige Methoden sind in Zeilen 33 aufgeführt.
   \item Die Member (Real- und Imaginärteil) werden in Zeilen 36-37 vom Typ \texttt{double} deklariert.
   \item \textbf{Zeilen 7,34,35}: Für Member und Methoden werden Zugriffsrechte vergeben:
   \begin{itemize}
      \item Mit \emph{public} gekennzeichnte Member und Methoden sind von außerhalb der Klasse sichtbar und benutzbar.
      \item  \emph{private} Member und Methoden sind nur innerhalb der Klasse sichtbar und benutzbar.
      \item Sind Member und Methoden als \emph{protected} gekennzeichnet, dann sind diese nur innerhalb der Klasse, oder von ihr abgeleiteter Klassen sichtbar und benutzbar (im Listing über einne leeren Menge).
      \item Aus Gründen der \emph{Datenkapselung} sollen Member nicht public sein. Um auf diese
       zugreifen zu können, müssen Getter-/Settermethoden von der Klasse bereitgestellt werden.
   \end{itemize}
   \item \textbf{Zeilen 9, 15, 20}: Deklaration des Standardkonstruktors (leere Parameterliste),
     eines Parameterkonstruktors und des Kopierkonstruktors.
   \item \textbf{Zeile 22}: Deklaration des Destruktors.
   \item \textbf{Zeile 26}: Deklaration des Zuweisungsoperators \texttt{operator=}\;.
   \item \textbf{Zeile 33}: Deklaration des Additionsoperators \texttt{operator+}\;.
     Das Schlüsselwort \texttt{const} am Ende der Deklaration garantiert, daß diese Methode
     keine Member der aufrufenden Instanz verändert. Damit ist diese Methode für kontante
     Instanzen anwendbar.
   \item Die Klasse ist für die Verarbeitung mit doygen dokumentiert.
\end{itemize}

%
\section{Konstruktoren}
\label{p:9.2}
Konstruktoren dienen der Erstinitialisierung von Membern einer Klasse und sie haben keinen
Rückgabetyp.
Man unterscheidet im wesentlichen zwischen dem Standardkonstruktor mit leerer Parameterliste,
den Parameterkonstruktoren und den Kopierkonstruktoren.
Ob diese Konstruktoren zur Funktionalität einer Klasse dazugehören sollen oder
nicht liegt im Ermessen des Designers der Klasse.

Deren Deklarationen erfolgt im Headerfile \emph{komplex.h}, die Implemenierungen schreiben wir hier
in das Sourcefile \emph{komplex.cpp}.
%
%

\subsection{Standardkonstruktor}
\label{p:9.2.1}
%
Damit eine Instanz über \verb|  Komplex c; | deklariert werden kann, ist
der Standardkonstruktor\index{Standardkonstruktor} notwendig.
%
\includecode[linerange={8-10,12-12,81-81},label=lst:komplex_komplex.h_b]{komplex/komplex.h}
{Deklaration des Standardkonstruktors im Headerfile}
%
Wir implementieren die Methode im Sourcefile.
\includecode[linerange={3-3,7-11},label=lst:komplex_komplex.cpp_b]{komplex/komplex.cpp}
{Implementierung des Standardkonstruktors im Sourcefile}
%
Jede Methode ist ein Funktion, welche genau einer Klasse zugeordnet ist.
Damit muß der Klassenname zur Identifikation mit in die Signatur\index{Signatur} der Funktion aufgenommen werden.
Damit hat der Standardkonstruktor der Klasse \texttt{Komplex} die Signatur
\verb| Komplex::Komplex( ) | welche im Sourcefile so angegeben werden muß.

Klassenmember sollten in der \emph{Member Initialization List}\index{Member Initialization List}
(Zeile~3 in Listing~\ref{lst:komplex_komplex.cpp_b}) initialisiert werden.
Für konstante Member und Basisklassen einer Klasse ist dies die einzige
Möglichkeit der Initialisierung.
Damit bleibt der Funktionskörper bei uns leer, was aber bei komplizierteren
Klassen nicht mehr der Fall sein muß.

\subsection{Parameterkonstruktor}
\label{p:9.2.2}
%
Eine Instanzdeklaration  \verb|  Komplex a(3.2,-1.1); | erfordert
den entsprechenden Parameterkonstruktor\index{Parameterkonstruktor}.
\includecode[linerange={8-10,17-18,81-81},label=lst:komplex_komplex.h_c]{komplex/komplex.h}
{Deklaration des Parameterkonstruktors im Headerfile}
%
Die Member werden in der Implementierung wieder über die {Member Initialization List}
deklariert
\includecode[linerange={3-3,13-17},label=lst:komplex_komplex.cpp_c]{komplex/komplex.cpp}
{Implementierung des Parameterkonstruktors im Sourcefile}
%
Durch das optionale zweite Argument der Deklaration in Listing~\ref{lst:komplex_komplex.h_c}
kann der Konstruktor auch als \verb| Komplex aa(-7.8) |  verwendet werden, wodurch
eine komplexe Zahl mit Imaginärteil gleich~0 festgelegt wird.

Dieser Konstruktor wird auch benutzt, um ein Casting\index{Casting}
von \texttt{double} zu \texttt{Komplex} durchführen wie in Zeile~17 von
Listing~\ref{lst:komplex_main.cpp}.
Dies kann notwendig werden, falls einer Funktion eine \texttt{double}-Variable
übergeben wird obwohl eine \texttt{Komplex}-Variable erwartet wird.
Diese implizite Nutzung des Konstruktor kann durch das Schlüsselwort
\texttt{explicit} in der Deklaration verhindert werden.

Eine Klasse kann mehrere Parameterkonstruktoren haben, selbstverständlich
mit unterschiedlichen Parameterlisten.

\subsection{Kopierkonstruktor}
\label{p:9.2.3}
%
Eine Deklaration  \verb|  Komplex b(a); | erfordert
den entsprechenden Kopierkonstruktor\index{Kopierkonstruktor}.
\includecode[linerange={8-10,24-24,81-81},label=lst:komplex_komplex.h_d]{komplex/komplex.h}
{Deklaration des Kopierkonstruktor im Headerfile}
%
Die Member werden in der Implementierung wieder über die {Member Initialization List}
deklariert
\includecode[linerange={3-3,19-23},label=lst:komplex_komplex.cpp_d]{komplex/komplex.cpp}
{Implementierung des Kopierkonstruktors im Sourcefile}
%
Der Zugriff auf private Member (\verb| org._re |) ist hier erlaubt, da die Methode
(also unser Konstruktor) zur gleichen Klasse gehört wie die Instanz~\texttt{org}.

Der Kopierkonstruktor ist ein spezieller Parameterkonstruktor einer Instanz der Klasse als
einziges Element der Parameterliste. Unser Kopierkonstruktor wird auch benutzt, wenn eine Instanz
der Klasse \texttt{Komplex} per value (also als Kopie!) an eine Funktion/Methode übergeben wird.
Wird die Instanz dagegen per reference übergeben, dann ist keinerlei Kopieroperation notwendig!
%
%
%
\section{Der Destruktor}
\label{p:9.3}
%
Der Destruktor\index{Destruktor} wird automatisch aufgerufen wenn das Ende des Gültigkeitsbereiches
einer Instanz dieser Klasse erreicht wird (schließende Klammer~$\}$~).
Diese Methode hat keinen Rückgabetyp.
Die Funktion des Destruktors besteht darin, die von der Instanz belegten Ressourcen
ordnungsgemäß freizugeben. Es gibt im Code stets soviele Desktruktoraufrufe wie es in Summe
Konstruktoraufrufe gibt.
Der Destruktor kann nicht direkt aufgerufen werden.
%
\includecode[linerange={8-10,27-27,81-81},label=lst:komplex_komplex.h_e]{komplex/komplex.h}
{Deklaration des Destruktors im Headerfile}
%
Falls der Destruktor automatisch vom Compiler generiert wird oder, wie bei uns,
einen leeren Funktionskörper enthält, dann werden die Destruktoren der Member (und der Basisklassen)
aufgerufen. Das Schlüsselwort \verb|virtual| sollte bei Klassenhierarchien
zum Destruktor hinzugefügt werden um die korrekte Freigabe aller Ressourcen der Hierarchie
zu garantieren.
\includecode[linerange={3-3,25-28},label=lst:komplex_komplex.cpp_e]{komplex/komplex.cpp}
{Implementierung des Destruktors im Sourcefile}
%
Bei eigenem dynamischen Resourcenmanagement muß der Destruktor diese Ressourcen freigeben.
Ansonsten wird Speicher nicht wieder freigegeben und irgendwann sind selbts 64~GB voll.

%
\section{Der Zuweisungsoperator}
\label{p:9.4}
Eine Zuweisung \verb| c = d; | wird vom Compiler als \verb|c.operator=(d)| umgesetzt,
es wird also eine Methode \verb| Komplex::operator= | in unserer Klasse benötigt.
Dieser Zuweisungsoperator ist die erste \glqq normale\grqq\  Methode, d.h., mit Rückgabetyp,
unserer Klasse.
%
\includecode[linerange={8-10,33-33,81-81},label=lst:komplex_komplex.h_f]{komplex/komplex.h}
{Deklaration des Zuweisungsoperators im Headerfile}
%
Die Implementierung der Zuweisung bei unserer Klasse ist nicht schwer.
Der Test \verb| if ( this!=&rhs)  | verhindert die Selbstzuweisung im Falle von \verb| c=c |,
mit \texttt{this} als dem Pointer auf die aktuelle Instanz
(bei uns also~\texttt{c} und somit ist hier \verb|this==&c|).{}
%
\includecode[linerange={3-3,30-38},label=lst:komplex_komplex.cpp_f]{komplex/komplex.cpp}
{Implementierung des Zuweisungsoperators im Sourcefile}
%
Die dreifache Verwendung von \verb|Komplex| in Zeile~2 mag verwirrend erscheinen, aber diese
bezeichnen Rückgabetyp, Klassenname und Typ des Inputparameters.
Der Rückgabetyp \verb|Komplex&| erfordert die Rückgabe einer Referenz auf die aktuelle Instanz
(wieder~\texttt{c}), realsiert in Zeile 9.
Damit sind Mehrfachzuweisungen wie \verb| f = c = d; |, aber auf Funktionsaufrufe wie
\verb| print(c=d); | möglich.

Es können auch mehrere Zuweisungsoperatoren mit anderen Klassen als Parameter
definiert werden. Mit jedem neuen Zuweisungsoperator erfolgt eine
\emph{Operatorüberladung}\index{Operatorüberladung}, d.h., gleicher Operator aber andere
Parameterliste.

%
\section{Compilergenerierte Methoden}
\label{p:9.5}
Folgende Methoden werden vom Compiler {automatisch} generiert wenn diese
in der Klassendeklaration nicht aufscheinen:
\begin{itemize}
    \item Kopierkonstruktor und Movekonstruktor (mit der eigenen Klasse)
    \item Zuweisungsoperator und Movezuweisung (mit der eigenen Klasse)
    \item Destruktor
\end{itemize}
Solange man in seiner Klasse nur einfache Datentypen, Klassen oder Container der STL
für die Member benutzt sind die compilergenerierten Methoden völlig ausreichend.
Auch eigene Klassen können dabei als Typ von Membern fungieren solange diese die
notwendigen Methoden besitzen. 
\\
Es empfiehlt sich die standardmäßig generierten Methode per \verb|=default| zu kennzeichnen, 
siehe \ghref{https://en.cppreference.com/w/cpp/language/rule_of_three}{\emph{Rule of Five}}.
\index{Rule of Five}\index{=default}
%
\begin{lstlisting}[caption=Rule of Five: explizit alles default,label=lst:9_5_1,basicstyle=\scriptsize]{}
class Komplex
{
public:
   // Rule of five
    Komplex(const Komplex&  org)            = default; // Copykonstruktor
    Komplex(      Komplex&& org)            = default; // Movekonstruktor
    Komplex& operator=(const Komplex&  rhs) = default; // Copy-Zuweisungsoperator
    Komplex& operator=(      Komplex&& rhs) = default; // Move-Zuweisungsoperator
    ~Komplex()                              = default; // Destruktor
}
\end{lstlisting}

Sobald Sie eine eigene dynamische Speicherverwaltung Ihrer Member verwenden wollen
(\verb|new|/\verb|delete[]| oder \verb|malloc|/\verb|free|), dann müssen
Sie obige  Methoden selbst implementieren (deep copy vs.\  shallow copy)!

Will man obige (oder andere) Methoden 
\ghref{https://en.cppreference.com/w/cpp/language/function\#Deleted_functions}{explizit verbieten}, 
dann ist der Deklaration ein \verb| = delete; | anzuschließen.
Ein generelles Verbot unseres Kopierkonstruktors wäre also im Headerfile
via \verb|  Komplex(const Komplex& org) = delete;| zu erreichen.
\index{=delete}

%http://www.developerfusion.com/article/133063/constructors-in-c11/
%http://de.wikibooks.org/wiki/C%2B%2B-Programmierung/_Eigene_Datentypen_definieren/_Leere_Klassen%3F#Compilergenerierte_Funktionen_explizit_verbieten

%
\section{Zugriffsmethoden}
\label{p:9.6}
Wenn die Member einer Klasse alle public wären, dann könnte man auf diese z.B.,
über \verb| c._re | direkt zugreifen. Jede Änderung der Membernamen oder
deren Speicherung würde ein Adaptieren aller Codes mit direktem Zugriff nach sich ziehen.
Daher sollen \textbf{Member immer private} sein und nur über Zugriffsmethoden
der Klasse erreichbar sein.

Diese, oft als Getter-/Settermethoden bezeichneten, Zugriffsmethoden werden aus
Effizienzgründen normalerweise gleich als Inlinemethoden in der Klassendeklaration
implementiert. Das folgende Listing zeigt die Setter- (schreibender Zugriff) und
Gettermethode (lesender Zugriff) für den Realteil,
die Methoden für den Imaginärteil sind analog.
Die Getter-/Settermethoden können beliebig bezeichnet werden.

\includecode[linerange={8-10,39-43,48-51,81-81},label=lst:komplex_komplex.h_g]{komplex/komplex.h}
{Deklaration und Implemenierung de Getter/Setter im Headerfile}

Das Listing~\ref{lst:demo_zugriff} zeigt den Einsatz der Getter-/Settermethoden.
Insbesondere ist bei der Deklaration der Gettermethoden (also beim Lesen) darauf zu achten,
daß diese Methoden als \verb|const| gekennzeichnet werden.
Ansonsten können im gesamten Code keinerlei konstante Instanzen der Klasse verwendet werden.
\begin{lstlisting}[caption={Demonstration der Getter/Setter},label=lst:demo_zugriff,
basicstyle=\scriptsize,numbers=left, numberstyle=\tiny, stepnumber=2, numbersep=5pt]
{
  Komplex d(3.2, -1);
  double dd = d.Get_re();             // get real part of 'd'
  d.Set_re(5.2);                      // new value for  real part of 'd'

  const Komplex b(-1.2, 3);           // 'b' is a constant instance
  cout << b.Get_re();                 //  ==> only constant methods allowed for 'b'
}
\end{lstlisting}
%
%
%
\section{Der Additionsoperator}
\label{p:9.7}
In Listing~\ref{lst:komplex_main.cpp} benötigen wir mehrfach den Additionsoperator ($+$), welchen
wir durch \emph{Operatorüberladung}\index{Operatorüberladung} für unsere
Klasse \texttt{Komplex} nutzbar machen können.

Die Operation \verb| a+b |  kann vom Compiler als \verb| a.operator+(b) | interpretiert werden
wodurch eine entsprechende Methode \verb|Komplex::operator+| mit entsprechendem Argument
in der Klasse notwendig wird.
%
\includecode[linerange={8-10,70-75,81-81},label=lst:komplex_komplex.h_h]{komplex/komplex.h}
{Deklaration des Additionsoperators im Headerfile}
%
Die Addition zweier komplexer Zahlen  erfolgt durch die Addition der entsprechenden
Real- und Imaginärteile.
Beide Summanden bleiben unverändert, damit ist diese Methode \texttt{const}.
%
\includecode[linerange={3-3,40-44},label=lst:komplex_komplex.cpp_h]{komplex/komplex.cpp}
{Implementierung der Methode zum Addieren}
%
Der Rückgabetyp des Additonsoperators darf keine Referenz sein,
da diese auf die temporäre Instanz~\texttt{tmp} verweist welche
nach dem Verlassen des Scopes der Methode nicht mehr existiert.
Der Funktionskörper der Methode kann auch ganz kompakt als
\verb|return Komplex(_re+rhs._re, _im+rhs._im);| geschrieben werden.

Zeile~5 im Listing~\ref{lst:demo_addition} zeigt, daß der Additionsoperator unserer Klasse, im Verbund
mit dem Parameterkonstruktor, auch die Addition mit einer \texttt{double}-Zahl als
zweitem Summanden erlaubt.
\begin{lstlisting}[caption={Additionsoperator erweitert},label=lst:demo_addition,
basicstyle=\scriptsize,numbers=left, numberstyle=\tiny, stepnumber=2, numbersep=5pt]
{
  Komplex a(3.2, -1), b;
  double dd = -2.1;

  c = a + dd;              // a.operator+( Komplex(dd,0.0) )
  c = dd + a;              // operator+(const double, const Komplex&)
}
\end{lstlisting}
%
Dies gelingt aber nicht mehr wenn der erste Summand eine \texttt{double}-Zahl ist, da
dann die hier nicht mögliche  Methode \verb|double::operator+| benötigt würde.
Der Ausweg besteht in einer Funktion (keine Methode!)
\verb|operator+(const double, const Komplex&)| welche die Summanden vertauscht und dann
die Methode aus unserer Klasse aufruft.
%
\includecode[linerange={3-3,55-60},label=lst:komplex_komplex.cpp_i]{komplex/komplex.cpp}
{Implementierung der Funktion zum Addieren mit \texttt{double}
 als erstem Summanden in Sourcefile}
%
Als Kommentar ist eine andere Möglichkeit der Implementierung angegeben.

\underline{Bemerkung:} Eine allgemeinere Lösung der Implementierung für vertauschbare
Operatoren arbeitet mit der Methode \verb|operator+=|
und der Funktion \verb|operator+(const Komplex&, const Komplex&)|.
%
%
%
\section{Der Ausgabeoperator}
\label{p:9.8}
Um eine Klasseninstanz einfach via \verb|cout << a|  auszugeben ist die
Funktion (keine Methode!) des Ausgabeoperators \verb|operator<<| erforderlich, welche
hierzu für eine Instanz unserer Klasse überladen wird.
%
\includecode[linerange={3-4,85-89},label=lst:komplex_komplex.j_f]{komplex/komplex.h}
{Deklaration des Ausgabeoperators im Headerfile}
%
Beim Ausgabeoperator ist der Rückgabetype wie auch der erste Parameter mit \verb|ostream&|
(Outputstream) festgelegt. Da \verb|operator<<| eine Funktion ist darf diese
nicht auf direkt auf die privaten Member der Klasse zugreifen, sondern
nur über die Zugriffsmethoden.
%
\includecode[linerange={1-3,49-53},label=lst:komplex_komplex.cpp_j]{komplex/komplex.cpp}
{Implementierung des Ausgabeoperators im Sourcefile}
%

\underline{Bemerkung:} Der Mechanismus der Datenkapselung
(nur eigene Methoden dürfen auf Member zugreifen) läßt sich über \texttt{friend}-Funktionen
\index{friend} aufweichen. Dies wird im Rahmen dieser LV nicht empfohlen.
%
%\begin{lstlisting}[caption={Ausgabeoerator als friend-Funktion},label=lst:demo_friend,
%basicstyle=\scriptsize,numbers=left, numberstyle=\tiny, stepnumber=2, numbersep=5pt]
%// Deklaration
%class Komplex
%{
%public:
     %...
     %friendo stream& operator<<(ostream& s, const Komplex& rhs);
%};

%// Implementierung
%ostream& operator<<(ostream& s, const Komplex& rhs)
%{
    %s << "("<< rhs._re<< ","<<rhs._im <<")";   // direct access to private members
    %return s;
%}
%\end{lstlisting}

%



%\section{Entwurf einer Klasse \texttt{StuKomplex}}
%\label{p:9.9}
%Beispiel V_{14}/v_{7a}






